\documentclass{ctexart}
\usepackage{avanti-color}
\usepackage{avanti-font}
\usepackage{avanti-math}
\usepackage{avanti-theorem}
\usepackage{avanti-others}

\tikzset{font=\large}
\tikzset{base/.style={smooth, very thick, Solarized-base03}}
\tikzset{point/.style={circle, minimum height=0.8cm, base, draw=Solarized-base03, fill=Solarized-base2}}
\tikzset{arrow/.style={->, -{Stealth[scale=0.8]}, base}}

\arraycolsep=4.0pt

\begin{document}
\title{\bf{最大流、线性规划、单纯形法}}
\author{张腾}
\date{}
\maketitle

本文给出用单纯形法求解如下流网络最大流的详细过程。
\begin{figure}[h]
    \centering
    \begin{tikzpicture}

        \pgfmathsetmacro{\l}{3.5};

        \node [point] (s) at (0,0) {$\sv$};
        \path (s) ++(30:\l)  node[point] (v1) {$\vv_1$};
        \path (s) ++(330:\l)  node[point] (v2) {$\vv_2$};
        \path (v1) ++(\l,0)  node[point] (v3) {$\vv_3$};
        \path (v2) ++(\l,0)  node[point] (v4) {$\vv_4$};
        \path (v4) ++(30:\l)  node[point] (t) {$\tv$};

        \draw [arrow] (s) -- (v1) node [above=0pt, sloped, pos=0.5] {$16$};
        \draw [arrow] (s) -- (v2) node [below=0pt, sloped, pos=0.4] {$13$};
        \draw [arrow] (v2) -- (v1) node [above=0pt, sloped, pos=0.5] {$4$};
        \draw [arrow] (v1) -- (v3) node [above=0pt, pos=0.5] {$12$};
        \draw [arrow] (v2) -- (v4) node [below=0pt, pos=0.5] {$14$};
        \draw [arrow] (v4) -- (v3) node [above=0pt, sloped, pos=0.5] {$7$};
        \draw [arrow] (v3) -- (v2) node [above=0pt, sloped, pos=0.5] {$9$};
        \draw [arrow] (v3) -- (t) node [above=0pt, sloped, pos=0.5] {$20$};
        \draw [arrow] (v4) -- (t) node [below=0pt, sloped, pos=0.5] {$4$};

    \end{tikzpicture}
\end{figure}

先将其转化成线性规划问题,根据容量限制和流量守恒分别有
\begin{align*}
    \begin{cases}
        0 \le x_1 \le 16 \\
        0 \le x_2 \le 13 \\
        0 \le x_3 \le 4  \\
        0 \le x_4 \le 12 \\
        0 \le x_5 \le 9  \\
        0 \le x_6 \le 14 \\
        0 \le x_7 \le 7  \\
        0 \le x_8 \le 20 \\
        0 \le x_9 \le 4
    \end{cases} \qquad \qquad
    \begin{cases}
        v_1: ~ x_1 + x_3 - x_4 = 0       \\
        v_2: ~ x_2 + x_5 - x_3 - x_6 = 0 \\
        v_3: ~ x_4 + x_7 - x_5 - x_8 = 0 \\
        v_4: ~ x_6 - x_7 - x_9 = 0
    \end{cases}
\end{align*}
为每个容量上限约束$x_i \le c_i$引入非负松弛变量$y_i$将其转化为等式约束$x_i + y_i = c_i$,于是最大流问题对应的标准形式的线性规划为
\begin{align*}
    \max \quad & x_1 + x_2                   \\
    \st \quad  & x_1 + x_3 - x_4 = 0         \\
               & x_2 + x_5 - x_3 - x_6 = 0   \\
               & x_4 + x_7 - x_5 - x_8 = 0   \\
               & x_6 - x_7 - x_9 = 0         \\
               & x_1 + y_1 = 16              \\
               & x_2 + y_2 = 13              \\
               & x_3 + y_3 = 4               \\
               & x_4 + y_4 = 12              \\
               & x_5 + y_5 = 9               \\
               & x_6 + y_6 = 14              \\
               & x_7 + y_7 = 7               \\
               & x_8 + y_8 = 20              \\
               & x_9 + y_9 = 4               \\
               & x_i, y_i \ge 0, ~ i \in [9]
\end{align*}
其中共有$18$个变量、$13$个等式约束,因此基本变量有$13$个,非基本变量有$5$个。

初始不妨取$x_{\{1, 2, 4, 5, 7\}}$为非基本变量,将基本变量$x_{\{3, 6, 8, 9\}}$和$y_{\{1, \ldots, 9\}}$由$x_{\{1, 2, 4, 5, 7\}}$表出:
\begin{align*}
    \begin{array}{rclcl}
        x_3 = -x_1 + x_4      & \Rightarrow & x_1 + x_3 - x_4 = 0                    & \Rightarrow & -x_1 + x_4 + y_3 = 4                  \\
        x_8 = x_4 - x_5 + x_7 & \Rightarrow & -x_4 + x_5 - x_7 + x_8 = 0             & \Rightarrow & x_4 - x_5 + x_7 + y_8 = 20            \\
        x_6 = x_2 + x_5 - x_3 & \Rightarrow & -x_1 - x_2 + x_4 - x_5 + x_6 = 0       & \Rightarrow & x_1 + x_2 - x_4 + x_5 + y_6 = 14      \\
        x_9 = x_6 - x_7       & \Rightarrow & -x_1 - x_2 + x_4 - x_5 + x_7 + x_9 = 0 & \Rightarrow & x_1 + x_2 - x_4 + x_5 - x_7 + y_9 = 4 \\
    \end{array}
\end{align*}
单纯形表为
\begin{align*}
    \begin{array}{c|cccccccccccccccccc:c}
            & x_1 & x_2 & x_3 & x_4 & x_5 & x_6 & x_7 & x_8 & x_9 & y_1 & y_2 & y_3 & y_4 & y_5 & y_6 & y_7 & y_8 & y_9      \\ \hline
        x_3 & 1   &     & 1   & -1  &     &     &     &     &     &     &     &     &     &     &     &     &     &     & 0  \\
        x_6 & -1  & -1  &     & 1   & -1  & 1   &     &     &     &     &     &     &     &     &     &     &     &     & 0  \\
        x_8 &     &     &     & -1  & 1   &     & -1  & 1   &     &     &     &     &     &     &     &     &     &     & 0  \\
        x_9 & -1  & -1  &     & 1   & -1  &     & 1   &     & 1   &     &     &     &     &     &     &     &     &     & 0  \\
        y_1 & 1   &     &     &     &     &     &     &     &     & 1   &     &     &     &     &     &     &     &     & 16 \\
        y_2 &     & 1   &     &     &     &     &     &     &     &     & 1   &     &     &     &     &     &     &     & 13 \\
        y_3 & -1  &     &     & 1   &     &     &     &     &     &     &     & 1   &     &     &     &     &     &     & 4  \\
        y_4 &     &     &     & 1   &     &     &     &     &     &     &     &     & 1   &     &     &     &     &     & 12 \\
        y_5 &     &     &     &     & 1   &     &     &     &     &     &     &     &     & 1   &     &     &     &     & 9  \\
        y_6 & 1   & 1   &     & -1  & 1   &     &     &     &     &     &     &     &     &     & 1   &     &     &     & 14 \\
        y_7 &     &     &     &     &     &     & 1   &     &     &     &     &     &     &     &     & 1   &     &     & 7  \\
        y_8 &     &     &     & 1   & -1  &     & 1   &     &     &     &     &     &     &     &     &     & 1   &     & 20 \\
        y_9 & 1   & 1   &     & -1  & 1   &     & -1  &     &     &     &     &     &     &     &     &     &     & 1   & 4  \\ \hdashline
            & -1  & -1  &     &     &     &     &     &     &     &     &     &     &     &     &     &     &     &     & 0
    \end{array}
\end{align*}
注意$x_{\{3, 6, 8, 9\}}$和$y_{\{1, \ldots, 9\}}$对应的列构成单位阵,因此令$x_{\{1, 2, 4, 5, 7\}} = 0$可得基本可行解
\begin{align*}
    \left[
        \begin{array}{cccccccccccccccccc:c}
            x_1 & x_2 & x_3 & x_4 & x_5 & x_6 & x_7 & x_8 & x_9 & y_1 & y_2 & y_3 & y_4 & y_5 & y_6 & y_7 & y_8 & y_9 & o \\
            0   & 0   & 0   & 0   & 0   & 0   & 0   & 0   & 0   & 16  & 13  & 4   & 12  & 9   & 14  & 7   & 20  & 4   & 0
        \end{array} \right]
\end{align*}
取$x_2$为输入变量,$\theta_{y_2} = 13$、$\theta_{y_6} = 14$、$\theta_{y_9} = 4$,因此$y_9$为分离变量,做初等行变换更新单纯形表
\begin{align*}
    \begin{array}{c|cccccccccccccccccc:c}
            & x_1 & x_2 & x_3 & x_4 & x_5 & x_6 & x_7 & x_8 & x_9 & y_1 & y_2 & y_3 & y_4 & y_5 & y_6 & y_7 & y_8 & y_9      \\ \hline
        x_3 & 1   &     & 1   & -1  &     &     &     &     &     &     &     &     &     &     &     &     &     &     & 0  \\
        x_6 &     &     &     &     &     & 1   & -1  &     &     &     &     &     &     &     &     &     &     & -1  & 4  \\
        x_8 &     &     &     & -1  & 1   &     & -1  & 1   &     &     &     &     &     &     &     &     &     &     & 0  \\
        x_9 &     &     &     &     &     &     &     &     & 1   &     &     &     &     &     &     &     &     & 1   & 4  \\
        y_1 & 1   &     &     &     &     &     &     &     &     & 1   &     &     &     &     &     &     &     &     & 16 \\
        y_2 & -1  &     &     & 1   & -1  &     & 1   &     &     &     & 1   &     &     &     &     &     &     & -1  & 9  \\
        y_3 & -1  &     &     & 1   &     &     &     &     &     &     &     & 1   &     &     &     &     &     &     & 4  \\
        y_4 &     &     &     & 1   &     &     &     &     &     &     &     &     & 1   &     &     &     &     &     & 12 \\
        y_5 &     &     &     &     & 1   &     &     &     &     &     &     &     &     & 1   &     &     &     &     & 9  \\
        y_6 &     &     &     &     &     &     & 1   &     &     &     &     &     &     &     & 1   &     &     & -1  & 10 \\
        y_7 &     &     &     &     &     &     & 1   &     &     &     &     &     &     &     &     & 1   &     &     & 7  \\
        y_8 &     &     &     & 1   & -1  &     & 1   &     &     &     &     &     &     &     &     &     & 1   &     & 20 \\
        x_2 & 1   & 1   &     & -1  & 1   &     & -1  &     &     &     &     &     &     &     &     &     &     & 1   & 4  \\ \hdashline
            &     &     &     & -1  & 1   &     & -1  &     &     &     &     &     &     &     &     &     &     & 1   & 4
    \end{array}
\end{align*}
当前基本可行解为
\begin{align*}
    \left[
        \begin{array}{cccccccccccccccccc:c}
            x_1 & x_2 & x_3 & x_4 & x_5 & x_6 & x_7 & x_8 & x_9 & y_1 & y_2 & y_3 & y_4 & y_5 & y_6 & y_7 & y_8 & y_9 & o \\
            0   & 4   & 0   & 0   & 0   & 4   & 0   & 0   & 4   & 16  & 9   & 4   & 12  & 9   & 10  & 7   & 20  & 0   & 4
        \end{array} \right]
\end{align*}
对应的流网络为
\begin{figure}[h]
    \centering
    \begin{tikzpicture}

        \pgfmathsetmacro{\l}{3.5};

        \node [point] (s) at (0,0) {$\sv$};
        \path (s) ++(30:\l)  node[point] (v1) {$\vv_1$};
        \path (s) ++(330:\l)  node[point] (v2) {$\vv_2$};
        \path (v1) ++(\l,0)  node[point] (v3) {$\vv_3$};
        \path (v2) ++(\l,0)  node[point] (v4) {$\vv_4$};
        \path (v4) ++(30:\l)  node[point] (t) {$\tv$};

        \draw [arrow] (s) -- (v1) node [above=0pt, sloped, pos=0.5] {$16$};
        \draw [arrow] (s) -- (v2) node [below=0pt, sloped, pos=0.4] {$4/13$};
        \draw [arrow] (v2) -- (v1) node [above=0pt, sloped, pos=0.5] {$4$};
        \draw [arrow] (v1) -- (v3) node [above=0pt, pos=0.5] {$12$};
        \draw [arrow] (v2) -- (v4) node [below=0pt, pos=0.5] {$4/14$};
        \draw [arrow] (v4) -- (v3) node [above=0pt, sloped, pos=0.5] {$7$};
        \draw [arrow] (v3) -- (v2) node [above=0pt, sloped, pos=0.5] {$9$};
        \draw [arrow] (v3) -- (t) node [above=0pt, sloped, pos=0.5] {$20$};
        \draw [arrow] (v4) -- (t) node [below=0pt, sloped, pos=0.5] {$4/4$};

    \end{tikzpicture}
\end{figure}

\newpage

第$2$轮,取$x_7$为输入变量,$\theta_{y_2} = 9$、$\theta_{y_6} = 10$、$\theta_{y_7} = 7$、$\theta_{y_8} = 20$,因此$y_7$为分离变量,做初等行变换更新单纯形表
\begin{align*}
    \begin{array}{c|cccccccccccccccccc:c}
            & x_1 & x_2 & x_3 & x_4 & x_5 & x_6 & x_7 & x_8 & x_9 & y_1 & y_2 & y_3 & y_4 & y_5 & y_6 & y_7 & y_8 & y_9      \\ \hline
        x_3 & 1   &     & 1   & -1  &     &     &     &     &     &     &     &     &     &     &     &     &     &     & 0  \\
        x_6 &     &     &     &     &     & 1   &     &     &     &     &     &     &     &     &     & 1   &     & -1  & 11 \\
        x_8 &     &     &     & -1  & 1   &     &     & 1   &     &     &     &     &     &     &     & 1   &     &     & 7  \\
        x_9 &     &     &     &     &     &     &     &     & 1   &     &     &     &     &     &     &     &     & 1   & 4  \\
        y_1 & 1   &     &     &     &     &     &     &     &     & 1   &     &     &     &     &     &     &     &     & 16 \\
        y_2 & -1  &     &     & 1   & -1  &     &     &     &     &     & 1   &     &     &     &     & -1  &     & -1  & 2  \\
        y_3 & -1  &     &     & 1   &     &     &     &     &     &     &     & 1   &     &     &     &     &     &     & 4  \\
        y_4 &     &     &     & 1   &     &     &     &     &     &     &     &     & 1   &     &     &     &     &     & 12 \\
        y_5 &     &     &     &     & 1   &     &     &     &     &     &     &     &     & 1   &     &     &     &     & 9  \\
        y_6 &     &     &     &     &     &     &     &     &     &     &     &     &     &     & 1   & -1  &     & -1  & 3  \\
        x_7 &     &     &     &     &     &     & 1   &     &     &     &     &     &     &     &     & 1   &     &     & 7  \\
        y_8 &     &     &     & 1   & -1  &     &     &     &     &     &     &     &     &     &     & -1  & 1   &     & 13 \\
        x_2 & 1   & 1   &     & -1  & 1   &     &     &     &     &     &     &     &     &     &     & 1   &     & 1   & 11 \\ \hdashline
            &     &     &     & -1  & 1   &     &     &     &     &     &     &     &     &     &     & 1   &     & 1   & 11
    \end{array}
\end{align*}
当前基本可行解为
\begin{align*}
    \left[
        \begin{array}{cccccccccccccccccc:c}
            x_1 & x_2 & x_3 & x_4 & x_5 & x_6 & x_7 & x_8 & x_9 & y_1 & y_2 & y_3 & y_4 & y_5 & y_6 & y_7 & y_8 & y_9 & o  \\
            0   & 11  & 0   & 0   & 0   & 11  & 7   & 7   & 4   & 16  & 2   & 4   & 12  & 9   & 3   & 0   & 13  & 0   & 11
        \end{array} \right]
\end{align*}
对应的流网络为
\begin{figure}[h]
    \centering
    \begin{tikzpicture}

        \pgfmathsetmacro{\l}{3.5};

        \node [point] (s) at (0,0) {$\sv$};
        \path (s) ++(30:\l)  node[point] (v1) {$\vv_1$};
        \path (s) ++(330:\l)  node[point] (v2) {$\vv_2$};
        \path (v1) ++(\l,0)  node[point] (v3) {$\vv_3$};
        \path (v2) ++(\l,0)  node[point] (v4) {$\vv_4$};
        \path (v4) ++(30:\l)  node[point] (t) {$\tv$};

        \draw [arrow] (s) -- (v1) node [above=0pt, sloped, pos=0.5] {$16$};
        \draw [arrow] (s) -- (v2) node [below=0pt, sloped, pos=0.4] {$11/13$};
        \draw [arrow] (v2) -- (v1) node [above=0pt, sloped, pos=0.5] {$4$};
        \draw [arrow] (v1) -- (v3) node [above=0pt, pos=0.5] {$12$};
        \draw [arrow] (v2) -- (v4) node [below=0pt, pos=0.5] {$11/14$};
        \draw [arrow] (v4) -- (v3) node [above=0pt, sloped, pos=0.5] {$7/7$};
        \draw [arrow] (v3) -- (v2) node [above=0pt, sloped, pos=0.5] {$9$};
        \draw [arrow] (v3) -- (t) node [above=0pt, sloped, pos=0.5] {$7/20$};
        \draw [arrow] (v4) -- (t) node [below=0pt, sloped, pos=0.5] {$4/4$};

    \end{tikzpicture}
\end{figure}

\newpage

第$3$轮,取$x_4$为输入变量,$\theta_{y_2} = 2$、$\theta_{y_3} = 4$、$\theta_{y_4} = 12$、$\theta_{y_8} = 13$,因此$y_2$为分离变量,做初等行变换更新单纯形表
\begin{align*}
    \begin{array}{c|cccccccccccccccccc:c}
            & x_1 & x_2 & x_3 & x_4 & x_5 & x_6 & x_7 & x_8 & x_9 & y_1 & y_2 & y_3 & y_4 & y_5 & y_6 & y_7 & y_8 & y_9      \\ \hline
        x_3 &     &     & 1   &     & -1  &     &     &     &     &     & 1   &     &     &     &     & -1  &     & -1  & 2  \\
        x_6 &     &     &     &     &     & 1   &     &     &     &     &     &     &     &     &     & 1   &     & -1  & 11 \\
        x_8 & -1  &     &     &     &     &     &     & 1   &     &     & 1   &     &     &     &     &     &     & -1  & 9  \\
        x_9 &     &     &     &     &     &     &     &     & 1   &     &     &     &     &     &     &     &     & 1   & 4  \\
        y_1 & 1   &     &     &     &     &     &     &     &     & 1   &     &     &     &     &     &     &     &     & 16 \\
        x_4 & -1  &     &     & 1   & -1  &     &     &     &     &     & 1   &     &     &     &     & -1  &     & -1  & 2  \\
        y_3 &     &     &     &     & 1   &     &     &     &     &     & -1  & 1   &     &     &     & 1   &     & 1   & 2  \\
        y_4 & 1   &     &     &     & 1   &     &     &     &     &     & -1  &     & 1   &     &     & 1   &     & 1   & 10 \\
        y_5 &     &     &     &     & 1   &     &     &     &     &     &     &     &     & 1   &     &     &     &     & 9  \\
        y_6 &     &     &     &     &     &     &     &     &     &     &     &     &     &     & 1   & -1  &     & -1  & 3  \\
        x_7 &     &     &     &     &     &     & 1   &     &     &     &     &     &     &     &     & 1   &     &     & 7  \\
        y_8 & 1   &     &     &     &     &     &     &     &     &     & -1  &     &     &     &     &     & 1   & 1   & 11 \\
        x_2 &     & 1   &     &     &     &     &     &     &     &     & 1   &     &     &     &     &     &     &     & 13 \\ \hdashline
            & -1  &     &     &     &     &     &     &     &     &     & 1   &     &     &     &     &     &     &     & 13
    \end{array}
\end{align*}
当前基本可行解为
\begin{align*}
    \left[
        \begin{array}{cccccccccccccccccc:c}
            x_1 & x_2 & x_3 & x_4 & x_5 & x_6 & x_7 & x_8 & x_9 & y_1 & y_2 & y_3 & y_4 & y_5 & y_6 & y_7 & y_8 & y_9 & o  \\
            0   & 13  & 2   & 2   & 0   & 11  & 7   & 9   & 4   & 16  & 0   & 2   & 10  & 9   & 3   & 0   & 11  & 0   & 13
        \end{array} \right]
\end{align*}
对应的流网络为
\begin{figure}[h]
    \centering
    \begin{tikzpicture}

        \pgfmathsetmacro{\l}{3.5};

        \node [point] (s) at (0,0) {$\sv$};
        \path (s) ++(30:\l)  node[point] (v1) {$\vv_1$};
        \path (s) ++(330:\l)  node[point] (v2) {$\vv_2$};
        \path (v1) ++(\l,0)  node[point] (v3) {$\vv_3$};
        \path (v2) ++(\l,0)  node[point] (v4) {$\vv_4$};
        \path (v4) ++(30:\l)  node[point] (t) {$\tv$};

        \draw [arrow] (s) -- (v1) node [above=0pt, sloped, pos=0.5] {$16$};
        \draw [arrow] (s) -- (v2) node [below=0pt, sloped, pos=0.4] {$13/13$};
        \draw [arrow] (v2) -- (v1) node [above=0pt, sloped, pos=0.5] {$2/4$};
        \draw [arrow] (v1) -- (v3) node [above=0pt, pos=0.5] {$2/12$};
        \draw [arrow] (v2) -- (v4) node [below=0pt, pos=0.5] {$11/14$};
        \draw [arrow] (v4) -- (v3) node [above=0pt, sloped, pos=0.5] {$7/7$};
        \draw [arrow] (v3) -- (v2) node [above=0pt, sloped, pos=0.5] {$9$};
        \draw [arrow] (v3) -- (t) node [above=0pt, sloped, pos=0.5] {$9/20$};
        \draw [arrow] (v4) -- (t) node [below=0pt, sloped, pos=0.5] {$4/4$};

    \end{tikzpicture}
\end{figure}

\newpage

第$4$轮,取$x_1$为输入变量,$\theta_{y_1} = 16$、$\theta_{y_4} = 10$、$\theta_{y_8} = 11$,因此$y_4$为分离变量,做初等行变换更新单纯形表
\begin{align*}
    \begin{array}{c|cccccccccccccccccc:c}
            & x_1 & x_2 & x_3 & x_4 & x_5 & x_6 & x_7 & x_8 & x_9 & y_1 & y_2 & y_3 & y_4 & y_5 & y_6 & y_7 & y_8 & y_9      \\ \hline
        x_3 &     &     & 1   &     & -1  &     &     &     &     &     & 1   &     &     &     &     & -1  &     & -1  & 2  \\
        x_6 &     &     &     &     &     & 1   &     &     &     &     &     &     &     &     &     & 1   &     & -1  & 11 \\
        x_8 &     &     &     &     & 1   &     &     & 1   &     &     &     &     & 1   &     &     & 1   &     &     & 19 \\
        x_9 &     &     &     &     &     &     &     &     & 1   &     &     &     &     &     &     &     &     & 1   & 4  \\
        y_1 &     &     &     &     & -1  &     &     &     &     & 1   & 1   &     & -1  &     &     & -1  &     & -1  & 6  \\
        x_4 &     &     &     & 1   &     &     &     &     &     &     &     &     & 1   &     &     &     &     &     & 12 \\
        y_3 &     &     &     &     & 1   &     &     &     &     &     & -1  & 1   &     &     &     & 1   &     & 1   & 2  \\
        x_1 & 1   &     &     &     & 1   &     &     &     &     &     & -1  &     & 1   &     &     & 1   &     & 1   & 10 \\
        y_5 &     &     &     &     & 1   &     &     &     &     &     &     &     &     & 1   &     &     &     &     & 9  \\
        y_6 &     &     &     &     &     &     &     &     &     &     &     &     &     &     & 1   & -1  &     & -1  & 3  \\
        x_7 &     &     &     &     &     &     & 1   &     &     &     &     &     &     &     &     & 1   &     &     & 7  \\
        y_8 &     &     &     &     & -1  &     &     &     &     &     &     &     & -1  &     &     & -1  & 1   &     & 1  \\
        x_2 &     & 1   &     &     &     &     &     &     &     &     & 1   &     &     &     &     &     &     &     & 13 \\ \hdashline
            &     &     &     &     & 1   &     &     &     &     &     &     &     & 1   &     &     & 1   &     & 1   & 23
    \end{array}
\end{align*}
当前基本可行解为
\begin{align*}
    \left[
        \begin{array}{cccccccccccccccccc:c}
            x_1 & x_2 & x_3 & x_4 & x_5 & x_6 & x_7 & x_8 & x_9 & y_1 & y_2 & y_3 & y_4 & y_5 & y_6 & y_7 & y_8 & y_9 & o  \\
            10  & 13  & 2   & 12  & 0   & 11  & 7   & 19  & 4   & 6   & 0   & 2   & 0   & 9   & 3   & 0   & 1   & 0   & 23
        \end{array} \right]
\end{align*}
对应的流网络为
\begin{figure}[h]
    \centering
    \begin{tikzpicture}

        \pgfmathsetmacro{\l}{3.5};

        \node [point] (s) at (0,0) {$\sv$};
        \path (s) ++(30:\l)  node[point] (v1) {$\vv_1$};
        \path (s) ++(330:\l)  node[point] (v2) {$\vv_2$};
        \path (v1) ++(\l,0)  node[point] (v3) {$\vv_3$};
        \path (v2) ++(\l,0)  node[point] (v4) {$\vv_4$};
        \path (v4) ++(30:\l)  node[point] (t) {$\tv$};

        \draw [arrow] (s) -- (v1) node [above=0pt, sloped, pos=0.5] {$10/16$};
        \draw [arrow] (s) -- (v2) node [below=0pt, sloped, pos=0.4] {$13/13$};
        \draw [arrow] (v2) -- (v1) node [above=0pt, sloped, pos=0.5] {$2/4$};
        \draw [arrow] (v1) -- (v3) node [above=0pt, pos=0.5] {$12/12$};
        \draw [arrow] (v2) -- (v4) node [below=0pt, pos=0.5] {$11/14$};
        \draw [arrow] (v4) -- (v3) node [above=0pt, sloped, pos=0.5] {$7/7$};
        \draw [arrow] (v3) -- (v2) node [above=0pt, sloped, pos=0.5] {$9$};
        \draw [arrow] (v3) -- (t) node [above=0pt, sloped, pos=0.5] {$19/20$};
        \draw [arrow] (v4) -- (t) node [below=0pt, sloped, pos=0.5] {$4/4$};

    \end{tikzpicture}
\end{figure}

目标行所有元素均非负,因此当前解已为最优解,对应的流网络达到最大流。

\end{document}