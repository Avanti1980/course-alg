\documentclass{ctexart}
\usepackage{avanti-color}
\usepackage{avanti-font}
\usepackage{avanti-math}
\usepackage{avanti-theorem}
\usepackage{avanti-others}

\tikzset{font=\large}
\tikzset{base/.style={smooth, very thick, Solarized-base03}}
\tikzset{point/.style={circle, minimum height=0.8cm, base, draw=Solarized-base03, fill=Solarized-base2}}
\tikzset{arrow/.style={->, -{Stealth[scale=0.8]}, base}}

\begin{document}
\title{\bf{用线性规划单纯形法求解最大流问题}}
\author{张腾}
\date{}
\maketitle

本讲义给出用单纯形法求解如下流网络最大流的详细过程。

\begin{figure}[h]
    \centering
    \begin{tikzpicture}

        \pgfmathsetmacro{\l}{3};

        \node [point] (s) at (0,0) {$\sv$};
        \path (s) ++(30:\l)  node[point] (v1) {$\vv_1$};
        \path (s) ++(330:\l)  node[point] (v2) {$\vv_2$};
        \path (v1) ++(\l,0)  node[point] (v3) {$\vv_3$};
        \path (v2) ++(\l,0)  node[point] (v4) {$\vv_4$};
        \path (v4) ++(30:\l)  node[point] (t) {$\tv$};

        \draw [arrow] (s) -- (v1) node [above=0pt, sloped, pos=0.5] {$16$};
        \draw [arrow] (s) -- (v2) node [below=0pt, sloped, pos=0.4] {$13$};
        \draw [arrow] (v2) -- (v1) node [above=0pt, sloped, pos=0.5] {$4$};
        \draw [arrow] (v1) -- (v3) node [above=0pt, pos=0.5] {$12$};
        \draw [arrow] (v2) -- (v4) node [below=0pt, pos=0.5] {$14$};
        \draw [arrow] (v4) -- (v3) node [above=0pt, sloped, pos=0.5] {$7$};
        \draw [arrow] (v3) -- (v2) node [above=0pt, sloped, pos=0.5] {$9$};
        \draw [arrow] (v3) -- (t) node [above=0pt, sloped, pos=0.5] {$20$};
        \draw [arrow] (v4) -- (t) node [below=0pt, sloped, pos=0.5] {$4$};

    \end{tikzpicture}
    \label{fig: city}
\end{figure}

先将其转化成线性规划问题,根据容量限制有
\begin{align*}
    0 \le x_1 \le 16
\end{align*}

\end{document}