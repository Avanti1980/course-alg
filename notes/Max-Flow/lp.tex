\documentclass{ctexart}
\usepackage{avanti-color}
\usepackage{avanti-font}
\usepackage{avanti-math}
\usepackage{avanti-theorem}
\usepackage{avanti-others}

\everymath{\color{Solarized-magenta}}
\pagestyle{empty} % 没有页眉和页脚

\tikzset{font=\tiny}
\tikzset{base/.style = {smooth, draw=Solarized-base01, text=Solarized-magenta}}
\tikzset{arrow/.style={->, -{Stealth[scale=0.8]}, base}}

\begin{document}
\title{\bf{线性规划}}
\author{张腾}
\date{}
\maketitle

本文给出用单纯形法求解如下流网络最大流的详细过程。



\begin{tikzpicture}[scale=1.5]

    \draw [arrow] (-0.2,0) -- (2.3,0) node[above] {$x_1$};
    \draw [arrow] (0,-0.2) -- (0,1.2) node[right] {$x_2$};

    %\draw [base,dashed] (0,0) -- (1,0) -- (1,1) -- (0,1) -- cycle;
    \draw [base] (0,0) -- (1,0) -- (1,1/3) -- (0,2/3) -- cycle;
    \fill [Solarized-red,opacity=0.5] (0,0) -- (1,0) -- (1,1/3) -- (0,2/3) -- cycle;


    \draw [base] (2,-1/6) -- (-1/6,11/12);

    \draw [base] (2.2,-0.1) -- (-0.2,1.1);
    \path (1.8,-0.15) node[draw=none,left] () {$3x_1+6x_2=5$};
    \path (1.3,0.4) node[draw=none,right] () {$3x_1+6x_2=6$};

    %\draw [base,Solarized-blue,domain=-2:2] plot(\x,{1*(\x)^3+1*(\x)^2-2*\x+2});

    % \path (6.3, 0.8) node[draw=none,left,Solarized-yellow] () {偏差};
    % \path (6.3, 1.9) node[draw=none,left,Solarized-blue] () {方差};
    % \path (6.3, 4.5) node[draw=none,left,Solarized-red] () {泛化风险};
    % \path (3.5, 0) node[draw=none,below,Solarized-base01] () {训练程度};

\end{tikzpicture}

\end{document}