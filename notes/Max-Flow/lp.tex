\documentclass{ctexart}
\usepackage{avanti-color}
\usepackage{avanti-font}
\usepackage{avanti-math}
\usepackage{avanti-theorem}
\usepackage{avanti-others}

\tikzset{font=\tiny}
\tikzset{base/.style = {smooth, draw=Solarized-base01, text=Solarized-magenta}}
\tikzset{arrow/.style={->, -{Stealth[scale=0.8]}, base}}

\begin{document}
\title{\bf{线性规划}}
\author{张腾}
\date{}
\maketitle

线性规划(linear programming)是在一组线性等式或不等式的约束下,求线性目标函数最值的问题,许多现实问题都可以形式化为线性规划问题。

\begin{example} [分数背包问题]
    设背包承重量为$10$,各物品价值如下:
    \begin{align*}
        \begin{array}{r|cccc} \hline
               & 物品$1$ & 物品$2$ & 物品$3$ & 物品$4$ \\ \hline
            重量 & 4     & 7     & 5     & 3     \\
            价值 & 40    & 42    & 25    & 12    \\ \hline
        \end{array}
    \end{align*}
    现允许物品按比例取走部分,求最大装包方案。

    对$i \in [4]$,设物品$i$取走的比例为$x_i$,则分数背包问题可形式化为
    \begin{align*}
        \quad \max \quad & 40 x_1 + 42 x_2 + 25 x_3 + 12 x_4    \\
        \st       \quad  & 4 x_1 + 7 x_2 + 5 x_3 + 3 x_4 \le 10 \\
                         & 0 \le x_1, x_2, x_3, x_4 \le 1
    \end{align*}
\end{example}

\begin{example} [最大流]
    单点
\end{example}

其标准形式为
\begin{align*}
    \max_\xv \quad & \cv^\top \xv                    \\
    \st      \quad & \Av \xv = \bv, ~ \xv \ge \zerov
\end{align*}
其中
\begin{align*}
    \cv = \begin{bmatrix}
              c_1 \\ \vdots \\ c_n
          \end{bmatrix}, \quad
    \xv = \begin{bmatrix}
              x_1 \\ \vdots \\ x_n
          \end{bmatrix}, \quad
    \Av= \begin{bmatrix}
             a_{11} & \ldots & a_{1n} \\
             \vdots & \ddots & \vdots \\
             a_{m1} & \ldots & a_{mn}
         \end{bmatrix} = \begin{bmatrix}
                             \av_1 & \ldots & \av_n
                         \end{bmatrix}, \quad
    \bv = \begin{bmatrix}
              b_1 \\ \vdots \\ b_m
          \end{bmatrix}
\end{align*}
集合$\Omega = \{\xv \mid \Av \xv = \bv, ~ \xv \ge \zerov \}$称为可行域,其中的点称为可行解。不失一般性设$\rank(\Av) = m < n$,否则可行域为单点集或空集。

\begin{example} [非标准形式的线性规划都可等价转化为标准形式]
    考虑线性规划
    \begin{align*}
        \max \quad & x_2 - x_1                                   \\
        \st  \quad & 3 x_1 = x_2 - 5, ~ |x_2| \le 2, ~ x_1 \le 0
    \end{align*}
    为使不等式只约束变量非负,其余都是等式约束
    \begin{enumerate}
        \item 记$y_1 = -x_1$,则$x_1 \le 0$变成非负约束$y_1 \ge 0$
        \item 由于$x_2$本身没有非负约束,只能记$x_2 = y_2 - y_3$,其中$y_2 \ge 0 $、$y_3 \ge 0$
        \item 注意$|x_2| \le 2$等价于$-2 \le y_2 - y_3 \le 2$,引入松弛变量$y_4 \ge 0$、$y_5 \ge 0$有$y_2 - y_3 + y_4 = 2$、$-y_2 + y_3 + y_5 = 2$
    \end{enumerate}
    综上有
    \begin{align*}
        \max \quad & y_1 + y_2 - y_3         \\
        \st  \quad & 3 y_1 + y_2 - y_3 = 5   \\
                   & y_2 - y_3 + y_4 = 2     \\
                   & -y_2 + y_3 + y_5 = 2    \\
                   & y_{\{1,2,3,4,5\}} \ge 0
    \end{align*}
\end{example}








$\Av \in \Rbb^{m \times n}$,$\bv \ge \zerov$。的$a$的






\begin{tikzpicture}[scale=1.5]

    \draw [arrow] (-0.2,0) -- (2.3,0) node[above] {$x_1$};
    \draw [arrow] (0,-0.2) -- (0,1.2) node[right] {$x_2$};

    %\draw [base,dashed] (0,0) -- (1,0) -- (1,1) -- (0,1) -- cycle;
    \draw [base] (0,0) -- (1,0) -- (1,1/3) -- (0,2/3) -- cycle;
    \fill [Solarized-red,opacity=0.5] (0,0) -- (1,0) -- (1,1/3) -- (0,2/3) -- cycle;


    \draw [base] (2,-1/6) -- (-1/6,11/12);

    \draw [base] (2.2,-0.1) -- (-0.2,1.1);
    \path (1.8,-0.15) node[draw=none,left] () {$3x_1+6x_2=5$};
    \path (1.3,0.4) node[draw=none,right] () {$3x_1+6x_2=6$};

    %\draw [base,Solarized-blue,domain=-2:2] plot(\x,{1*(\x)^3+1*(\x)^2-2*\x+2});

    % \path (6.3, 0.8) node[draw=none,left,Solarized-yellow] () {偏差};
    % \path (6.3, 1.9) node[draw=none,left,Solarized-blue] () {方差};
    % \path (6.3, 4.5) node[draw=none,left,Solarized-red] () {泛化风险};
    % \path (3.5, 0) node[draw=none,below,Solarized-base01] () {训练程度};

\end{tikzpicture}

\end{document}