\documentclass{ctexart}
\usepackage{avanti-color}
\usepackage{avanti-font}
\usepackage{avanti-math}
\usepackage{avanti-theorem}
\usepackage{avanti-others}

\tikzset{font=\large}
\tikzset{base/.style={smooth, very thick, Solarized-base03}}
\tikzset{point/.style={circle, minimum height=0.8cm, base, draw=Solarized-base03, fill=Solarized-base2}}
\tikzset{arrow/.style={->, -{Stealth[scale=0.8]}, base}}

\begin{document}
\title{\bf{线性规划、单纯形法}}
\author{张腾\thanks{tengzhang@hust.edu.cn}}
%\affil{华中科技大学~计算机学院}
\date{\today}
\maketitle

线性规划是在一组线性等式或不等式的约束下,求线性目标函数最值的问题,现实中的许多问题都可化为线性规划问题。

\begin{example} [分数背包问题] \label{exam: bag}
    设背包承重量为$10$,各物品价值如下:
    \begin{align*}\arraycolsep=10.0pt
        \begin{array}{r|cccc} \hline
               & 物品$1$ & 物品$2$ & 物品$3$ & 物品$4$ \\ \hline
            重量 & 4     & 7     & 5     & 3     \\
            价值 & 40    & 42    & 25    & 12    \\ \hline
        \end{array}
    \end{align*}
    现允许物品按比例取走部分,求最大装包方案。

    对$i \in [4]$,设物品$i$取走的比例为$x_i$,可得如下线性规划
    \begin{align*}
        \max \quad & 40 x_1 + 42 x_2 + 25 x_3 + 12 x_4    \\
        \st \quad  & 4 x_1 + 7 x_2 + 5 x_3 + 3 x_4 \le 10 \\
                   & 0 \le x_1, x_2, x_3, x_4 \le 1
    \end{align*}
\end{example}

\begin{remark}
    如果不允许只取部分(0/1背包问题),约束$0 \le x_1, x_2, x_3, x_4 \le 1$将变成$x_1, x_2, x_3, x_4 \in \{0, 1\}$,此时问题就变成了整数线性规划,比线性规划要难得多。
\end{remark}

\begin{example} [最大流] \label{exam: flow}
    给定如下的流网络,求最大流。
    \begin{figure}[h]
        \centering
        \begin{tikzpicture}

            \pgfmathsetmacro{\l}{3.5};

            \node [point] (s) at (0,0) {$\sv$};
            \path (s) ++(30:\l)  node[point] (v1) {$\vv_1$};
            \path (s) ++(330:\l)  node[point] (v2) {$\vv_2$};
            \path (v1) ++(\l,0)  node[point] (v3) {$\vv_3$};
            \path (v2) ++(\l,0)  node[point] (v4) {$\vv_4$};
            \path (v4) ++(30:\l)  node[point] (t) {$\tv$};

            \draw [arrow] (s) -- (v1) node [above=0pt, sloped, pos=0.5] {$16$};
            \draw [arrow] (s) -- (v2) node [below=0pt, sloped, pos=0.4] {$13$};
            \draw [arrow] (v2) -- (v1) node [above=0pt, sloped, pos=0.5] {$4$};
            \draw [arrow] (v1) -- (v3) node [above=0pt, pos=0.5] {$12$};
            \draw [arrow] (v2) -- (v4) node [below=0pt, pos=0.5] {$14$};
            \draw [arrow] (v4) -- (v3) node [above=0pt, sloped, pos=0.5] {$7$};
            \draw [arrow] (v3) -- (v2) node [above=0pt, sloped, pos=0.5] {$9$};
            \draw [arrow] (v3) -- (t) node [above=0pt, sloped, pos=0.5] {$20$};
            \draw [arrow] (v4) -- (t) node [below=0pt, sloped, pos=0.5] {$4$};

        \end{tikzpicture}
    \end{figure}

    设$9$条边上的流量分别为$x_1, \ldots, x_9$,可得如下线性规划
    \begin{align*}
        \max \quad & x_1 + x_2                 \\
        \st \quad  & 0 \le x_1 \le 16          \\
                   & 0 \le x_2 \le 13          \\
                   & 0 \le x_3 \le 4           \\
                   & 0 \le x_4 \le 12          \\
                   & 0 \le x_5 \le 9           \\
                   & 0 \le x_6 \le 14          \\
                   & 0 \le x_7 \le 7           \\
                   & 0 \le x_8 \le 20          \\
                   & 0 \le x_9 \le 4           \\
                   & x_1 + x_3 - x_4 = 0       \\
                   & x_2 + x_5 - x_3 - x_6 = 0 \\
                   & x_4 + x_7 - x_5 - x_8 = 0 \\
                   & x_6 - x_7 - x_9 = 0
    \end{align*}
    其中前$9$个不等式约束对应容量限制,后$4$个等式约束对应流量守恒。
\end{example}

$\Rbb^2$中的线性规划只有$2$个变量,线性等式约束是一条直线,线性不等式约束是一个半平面,可采用图解法。

\begin{example} \label{exam: illustration}
    考虑如下线性规划
    \begin{align*}
        \max \quad & 3 x_1 + 5 x_2      \\
        \st  \quad & x_1 + 5 x_2 \le 40 \\
                   & 2 x_1 + x_2 \le 20 \\
                   & x_1 + x_2 \le 12   \\
                   & x_1,x_2 \ge 0
    \end{align*}

    先确定\blue{可行域},即满足所有约束的\blue{可行解}构成的集合。该例中共有$5$个线性不等式约束,每个对应一个半平面,因此可行域为$5$个半平面相交出的凸五边形(图\ref{fig: lp}中红色部分)。

    引入直线簇$y = 3 x_1 + 5 x_2$,其中不同的$y$对应不同的直线,这些直线都是平行的。先将$y$取为一个较大的值使直线与凸五边形不相交,然后逐渐减小$y$,这相当于从上向下平移直线$y = 3 x_1 + 5 x_2$使其逐渐靠近凸五边形,当其与凸五边形相切时,切点就是\blue{最优解},
    \begin{figure}[h]
        \centering
        \tikzset{font=\small}
        \begin{tikzpicture}[scale=0.7]

            \draw [arrow] (-0.5,0) -- (14,0) node[above] {$x_1$};
            \draw [arrow] (0,-0.5) -- (0,9.5) node[right] {$x_2$};

            \draw [base,thin,dashed] (-0.5,8.1) -- (10,6) node [above=0pt, sloped, pos=0.2] {$x_1 + 5 x_2 \le 40$};
            \draw [base,thin,dashed] (10.2,-0.4) -- (5,10) node [above=0pt, sloped, pos=0.85] {$2 x_1 + x_2 \le 20$};
            \draw [base,thin,dashed] (12.2,-0.2) -- (2,10) node [above=0pt, sloped, pos=0.12] {$x_1 + x_2 \le 12$};

            \draw [base] (1,9.4) -- (13,2.2) node [above=0pt, sloped, pos=0.85] {$3 x_1 + 5 x_2 = 50$};

            \draw [base] (0,0) -- (10,0) -- (8,4) -- (5,7) -- (0,8) -- cycle;
            \fill [Solarized-red,opacity=0.5] (0,0) -- (10,0) -- (8,4) -- (5,7) -- (0,8) -- cycle;

            \path (0, -0.4) node[draw=none,left] () {$(0,0)$};
            \path (9.6, 0) node[draw=none,below] () {$(10,0)$};
            \path (0, 7.6) node[draw=none,left] () {$(0,8)$};
            \path (8,4) node[draw=none,right] () {$(8,4)$};
            \path (5.3,7) node[draw=none,above] () {$(5,7)$};

        \end{tikzpicture}
        \caption{直线簇与可行域相切于最优解$(5,7)$处,目标函数最优值为$50$。}
        \label{fig: lp}
    \end{figure}
\end{example}

\section{标准型}

当变量多于$2$个时,图解法就不再适用了,需要更一般性的方法。对此,要先将问题转化为如下的标准型(不等式只约束变量非负,其余都是等式约束):
\begin{align*}
    \max \quad & \cv^\top \xv   \\
    \st  \quad & \Av \xv = \bv  \\
               & \xv \ge \zerov
\end{align*}
其中
\begin{align*}
    \cv = \begin{bmatrix}
              c_1 \\ \vdots \\ c_n
          \end{bmatrix}, \quad
    \xv = \begin{bmatrix}
              x_1 \\ \vdots \\ x_n
          \end{bmatrix}, \quad
    \Av= \begin{bmatrix}
             a_{11} & \ldots & a_{1n} \\
             \vdots & \ddots & \vdots \\
             a_{m1} & \ldots & a_{mn}
         \end{bmatrix} = \begin{bmatrix}
                             \av_1 & \ldots & \av_n
                         \end{bmatrix}, \quad
    \bv = \begin{bmatrix}
              b_1 \\ \vdots \\ b_m
          \end{bmatrix}
\end{align*}
不失一般性可设$\bv \ge \zerov$,若某个$b_i < 0$,对该约束两边取反即可。对一般形式的线性规划,可按以下步骤将其转化成标准型:
\begin{itemize}
    \item 对非正变量$x \le 0$,令$y = -x$作为替代;
    \item 对无约束变量$x$,将其表示成两个非负变量的差$x = u - v$;
    \item 对$\av^\top \xv \le b$型不等式约束,引入松弛变量$y \ge 0$将其转化为等式约束$\av^\top \xv + y = b$;
    \item 对$\av^\top \xv \ge b$型不等式约束,引入剩余变量$y \ge 0$将其转化为等式约束$\av^\top \xv - y = b$。
\end{itemize}

\begin{example}
    将如下线性规划转化为标准型
    \begin{align*}
        \max \quad & x_2 - x_1       \\
        \st  \quad & 3 x_1 = x_2 - 5 \\
                   & |x_2| \le 2     \\
                   & x_1 \le 0
    \end{align*}

    $x_1$非正,令$y_1 = -x_1 \ge 0$作为替代,$x_2$无约束,令$x_2 = y_2 - y_3$,其中$y_2 \ge 0 $、$y_3 \ge 0$,注意
    \begin{align*}
        |x_2| \le 2 \Longleftrightarrow
        \begin{cases}
            y_2 - y_3 \le 2 \Longleftrightarrow y_2 - y_3 + y_4 = 2, ~ y_4 \ge 0 \\
            -y_2 + y_3 \le 2 \Longleftrightarrow -y_2 + y_3 + y_5 = 2, ~ y_5 \ge 0
        \end{cases}
    \end{align*}
    于是可得标准型
    \begin{align*}
        \max \quad & y_1 + y_2 - y_3           \\
        \st  \quad & 3 y_1 + y_2 - y_3 = 5     \\
                   & y_2 - y_3 + y_4 = 2       \\
                   & -y_2 + y_3 + y_5 = 2      \\
                   & y_1,y_2,y_3,y_4,y_5 \ge 0
    \end{align*}
\end{example}

\section{基本解}

所有可行解都是线性方程组$\Av \xv = \av_1 x_1 + \cdots + \av_n x_n = \bv$的解,不失一般性可设$\rank(\Av) = m$,即$\Av$是行满秩矩阵,否则存在冗余约束。此外设$m < n$,即线性等式约束个数严格小于变量个数,否则可行域为单点集或空集。

从$\Av$的$n$个列中挑选$m$个线性无关的列作为\blue{基向量},不妨就取$\Av$的前$m$列,否则做列交换使前$m$列线性无关(列对应的$x$分量也要跟着交换),这样$\Av$可写成分块矩阵
\begin{align*}
    \Av = \begin{bmatrix} \Bv & \Dv \end{bmatrix}, \quad \Bv =
    \begin{bmatrix}
        \av_1 & \cdots & \av_m
    \end{bmatrix}, \quad \Dv =
    \begin{bmatrix}
        \av_{m+1} & \cdots & \av_n
    \end{bmatrix} \in \Rbb^{m \times (n-m)}
\end{align*}
其中$\Bv$是$m$阶可逆方阵。于是
\begin{align*}
    \bv = \Av \xv = \begin{bmatrix} \Bv & \Dv \end{bmatrix} 
    \begin{bmatrix} \xv_\Bv \\ \xv_\Dv \end{bmatrix} = \Bv \xv_\Bv + \Dv \xv_\Dv
\end{align*}
其中$\xv_\Bv$称为\blue{基变量},$\xv_\Dv$为\blue{非基变量}。令非基变量为零可得$\xv_\Bv = \Bv^{-1} \bv$,这就得到了$\Av \xv = \bv$在基$\Bv$下的\blue{基本解}:
\begin{align*}
    \xvhat_\Bv = \begin{bmatrix} \xv_\Bv \\ \xv_\Dv \end{bmatrix} = \begin{bmatrix} \Bv^{-1} \bv \\ \zerov \end{bmatrix}, \quad \Av \xvhat = \begin{bmatrix} \Bv & \Dv \end{bmatrix} \begin{bmatrix} \xv_\Bv \\ \zerov \end{bmatrix} = \Bv \xv_\Bv = \bv
\end{align*}
如果基本解还是线性规划的可行解(所有变量非负),则称为\blue{基本可行解}。

\begin{theorem} [线性规划基本定理]
    对于线性规划的标准型,有如下两个命题:
    \begin{enumerate}
        \item 如果存在可行解,则一定存在基本可行解;
        \item 如果存在最优可行解,则一定存在最优基本可行解。
    \end{enumerate}
\end{theorem}

\begin{proof}
    1. 设$\xv$是一个可行解并有$p$个正元素,不失一般性,可设前$p$个元素为正,于是
    \begin{align*}
        \Av \xv = \av_1 x_1 + \cdots + \av_p x_p = \bv
    \end{align*}
    此时分两种情况:
    \begin{itemize}
        \item $\av_1, \ldots, \av_p$线性无关,则$p \le m$。若$p = m$,$\xv$就是基本可行解;若$p < m$,从$\Av$的剩余列中挑选$m-p$个列与$\av_1, \ldots, \av_p$构成基,此时$\xv$就是对应该基的基本可行解。
        \item $\av_1, \ldots, \av_p$线性相关,可以去掉一些冗余列使其线性无关,从而转化为前一种情况。设不全为零的实数$y_1, \ldots, y_p$使得$\av_1 y_1 + \cdots + \av_p y_p = \zerov$且至少某个$y_i > 0$,否则对所有$y_i$取反即可,于是对任意$\epsilon$有
              \begin{align*}
                  \bv = \av_1 (x_1 - \epsilon y_1) + \cdots + \av_p (x_p - \epsilon y_p) = \Av (\xv - \epsilon \yv), \quad
                  \yv \triangleq \begin{bmatrix}
                                     y_1 \\ \vdots \\ y_p \\ \zerov
                                 \end{bmatrix}
              \end{align*}
              让$\epsilon$从$0$增大直到$\xv - \epsilon \yv$的前$p$个正分量出现$0$,即取$\epsilonhat = \min\{ x_i / y_i: y_i > 0, i \in [p] \}$,这样就得到了只有$p-1$个正分量的可行解,重复该操作直到正分量对应的列线性无关。
    \end{itemize}

    2. 设$\xv$是一个最优可行解且前$p$个元素为正,若$\av_1, \ldots, \av_p$线性无关,证明同命题1;若$\av_1, \ldots, \av_p$线性相关,可继续沿用命题1中去冗余列的方式,但还需证明对任意$\epsilon$,$\xv - \epsilon \yv$都是最优解,这只需证明$\cv^\top \yv = 0$。注意只要$|\epsilon| \le \min\{ | x_i / y_i |: y_i \ne 0, i \in [p] \}$,$\xv - \epsilon \yv$都是可行解,因此若$\cv^\top \yv \ne 0$,根据其符号总能取某个充分小的$\epsilon$使得$\cv^\top (\xv - \epsilon \yv) = \cv^\top \xv - \epsilon \cv^\top \yv > \cv^\top \xv$,这与$\xv$是最优解矛盾。
\end{proof}

根据该定理,线性规划的求解可转化为对基本可行解的搜索问题,依次对基本可行解的最优性进行检查即可。

\section{几何视角下的线性规划}

线性规划属于凸优化的范畴,线性目标函数显然是凸函数,可行域$\Omega = \{ \xv \mid \Av \xv = \bv, \xv \ge \zerov \}$是凸集,因为对$\forall \xv_1, \xv_2 \in \Omega$和$\forall \alpha \in (0,1)$有
\begin{align*}
    \Av (\alpha \xv_1 + (1 - \alpha) \xv_2) = \alpha \Av \xv_1 + (1 - \alpha) \Av \xv_2 = \alpha \bv + (1 - \alpha) \bv = \bv, \quad \alpha \xv_1 + (1 - \alpha) \xv_2 \ge \zerov
\end{align*}
即连接$\Omega$内任意两点的线段依然属于$\Omega$。对凸集$\Omega$中的点$\xv$,若它无法表示成$\Omega$中另外两点的凸组合,则称$\xv$为$\Omega$的\blue{极点},即
\begin{align*}
    \xv\text{是极点}, ~ \xv = \alpha \xv_1 + (1 - \alpha) \xv_2, ~ \alpha \in (0,1) \Longrightarrow \xv_1 = \xv_2 = \xv
\end{align*}

\begin{theorem}
    $\xv$是$\Omega = \{ \xv \mid \Av \xv = \bv, \xv \ge \zerov \}$的极点当且仅当$\xv$是$\Av \xv = \bv, \xv \ge \zerov$的基本可行解。
\end{theorem}

\begin{proof}
    待补充。
\end{proof}

\begin{example} \label{exam: simplex}
    再看例\ref{exam: illustration}中的线性规划:
    \begin{align*}
        \max \quad & 3 x_1 + 5 x_2      \\
        \st  \quad & x_1 + 5 x_2 \le 40 \\
                   & 2 x_1 + x_2 \le 20 \\
                   & x_1 + x_2 \le 12   \\
                   & x_1,x_2 \ge 0
    \end{align*}
    先转化为标准型,为$3$个线性不等式约束分别引入松弛变量$x_3$、$x_4$、$x_5$可得线性方程组
    \begin{align*}
        \begin{bmatrix}
            1 & 5 & 1 & 0 & 0 \\
            2 & 1 & 0 & 1 & 0 \\
            1 & 1 & 0 & 0 & 1
        \end{bmatrix}
        \begin{bmatrix}
            x_1 \\ x_2 \\ x_3 \\ x_4 \\ x_5
        \end{bmatrix} =
        \underbrace{\begin{bmatrix}
                            1 \\ 2 \\ 1
                        \end{bmatrix}}_{\av_1} x_1 +
        \underbrace{\begin{bmatrix}
                            5 \\ 1 \\ 1
                        \end{bmatrix}}_{\av_2} x_2 +
        \underbrace{\begin{bmatrix}
                            1 \\ 0 \\ 0
                        \end{bmatrix}}_{\av_3} x_3 +
        \underbrace{\begin{bmatrix}
                            0 \\ 1 \\ 0
                        \end{bmatrix}}_{\av_4} x_4 +
        \underbrace{\begin{bmatrix}
                            0 \\ 0 \\ 1
                        \end{bmatrix}}_{\av_5} x_5 =
        \underbrace{\begin{bmatrix}
                            40 \\ 20 \\ 12
                        \end{bmatrix}}_{\bv}
    \end{align*}
    显然取$\av_3$、$\av_4$、$\av_5$作为基向量,即令$x_3$、$x_4$、$x_5$作为基变量,可得基本可行解
    \begin{align*}
        40 \av_3 + 20 \av_4 + 12 \av_5 = \bv, \quad
        \left[ \begin{array}{ccccc}
                       x_1 & x_2 & x_3 & x_4 & x_5 \\
                       0   & 0   & 40  & 20  & 12
                   \end{array} \right]
    \end{align*}
    对应$\Rbb^2$中可行域的极点$[0,0]$,目标函数值$0 < 50$,因此还不是最优解。

    根据迭代改进的思路,需要从当前极点移动到邻近的可使目标函数值增大的极点。现选择$\av_1$作为新的基向量(入基)并移除原来的某个基向量(出基),注意$\av_1 = \av_3 + 2 \av_4 + \av_5$,于是
    \begin{align*}
        \epsilon \av_1 + (40 - \epsilon) \av_3 + (20 - 2 \epsilon) \av_4 + (12 - \epsilon) \av_5 = \bv, \quad
        \left[ \begin{array}{ccccc}
                       x_1      & x_2 & x_3           & x_4             & x_5           \\
                       \epsilon & 0   & 40 - \epsilon & 20 - 2 \epsilon & 12 - \epsilon
                   \end{array} \right]
    \end{align*}
    让$\epsilon$从$0$增大,$x_1$变成正数,$x_3$、$x_4$、$x_5$逐渐变小,当$\epsilon$增大到$10$时,$x_4$减小到$0$,即$\av_4$出基,得到一个新的基本可行解
    \begin{align*}
        10 \av_1 + 30 \av_3 + 2 \av_5 = \bv, \quad
        \left[ \begin{array}{ccccc}
                       x_1 & x_2 & x_3 & x_4 & x_5 \\
                       10  & 0   & 30  & 0   & 2
                   \end{array} \right]
    \end{align*}
    对应$\Rbb^2$中可行域的极点$[10,0]$,目标函数值$30 < 50$,依然不是最优解。

    重复前面的操作,现选择$\av_2$作为新的基向量,注意$\av_2 = \frac{1}{2} \av_1 + \frac{9}{2} \av_3 + \frac{1}{2} \av_5$,于是
    \begin{align*}
        \left( 10 - \frac{1}{2} \epsilon \right) \av_1 + \epsilon \av_2 + \left( 30 - \frac{9}{2} \epsilon \right) \av_3 + \left( 2 - \frac{1}{2} \epsilon \right) \av_5 = \bv, \quad
        \left[ \begin{array}{ccccc}
                       x_1                       & x_2      & x_3                       & x_4 & x_5                      \\
                       10 - \frac{1}{2} \epsilon & \epsilon & 30 - \frac{9}{2} \epsilon & 0   & 2 - \frac{1}{2} \epsilon
                   \end{array} \right]
    \end{align*}
    让$\epsilon$从$0$增大,$x_2$变成正数,$x_1$、$x_3$、$x_5$逐渐变小,当$\epsilon$增大到$4$时,$x_5$减小到$0$,即$\av_5$出基,得到一个新的基本可行解
    \begin{align*}
        10 \av_1 + 30 \av_3 + 2 \av_5 = \bv, \quad
        \left[ \begin{array}{ccccc}
                       x_1 & x_2 & x_3 & x_4 & x_5 \\
                       8   & 4   & 12  & 0   & 0
                   \end{array} \right]
    \end{align*}
    对应$\Rbb^2$中可行域的极点$[8,4]$,目标函数值$44 < 50$,依然不是最优解。

    重复前面的操作,现选择$\av_4$作为新的基向量,注意$\av_4 = \av_1 - \av_2 + 4 \av_3$,于是
    \begin{align*}
        (8 - \epsilon) \av_1 + (4 + \epsilon) \av_2 + (12 - 4 \epsilon) \av_3 + \epsilon \av_4 = \bv, \quad
        \left[ \begin{array}{ccccc}
                       x_1          & x_2          & x_3             & x_4      & x_5 \\
                       8 - \epsilon & 4 + \epsilon & 12 - 4 \epsilon & \epsilon & 0
                   \end{array} \right]
    \end{align*}
    让$\epsilon$从$0$增大,$x_4$变成正数,$x_1$、$x_3$逐渐变小($x_2$变大不用管,不会破坏非负约束),当$\epsilon$增大到$3$时,$x_3$减小到$0$,即$\av_3$出基,得到一个新的基本可行解
    \begin{align*}
        5 \av_1 + 7 \av_2 + 3 \av_4 = \bv, \quad
        \left[ \begin{array}{ccccc}
                       x_1 & x_2 & x_3 & x_4 & x_5 \\
                       5   & 7   & 0   & 3   & 0
                   \end{array} \right]
    \end{align*}
    对应$\Rbb^2$中可行域的极点$[5,7]$,目标函数值$50$,这就是最优解。
\end{example}

这种从一个极点转移到另一个极点,迭代改进的操作方式就是单纯形法求线性规划的基本思路。

\section{单纯形法}

设当前基向量为$\av_1, \ldots, \av_m$,待入基向量为$\av_q$,例\ref{exam: simplex}中每轮迭代都要将$\bv$和$\av_q$用当前基线性表出
\begin{align}
    \label{eq: coef1}
    \bv   & = y_{10} \av_1 + \cdots + y_{m0} \av_m \\
    \label{eq: coef2}
    \av_q & = y_{1q} \av_1 + \cdots + y_{mq} \av_m
\end{align}
令$\eqref{eq: coef1} - \epsilon \times \eqref{eq: coef2}$可得关于$\epsilon$的恒等式
\begin{align*}
    (y_{10} - \epsilon y_{1q}) \av_1 + \cdots + (y_{m0} - \epsilon y_{mq}) \av_m + \epsilon \av_q = \bv
\end{align*}
让$\epsilon$从$0$增大直到某个$\av_p$出基,其中$p = \argmin_i \{ y_{i0} / y_{iq} : y_{iq} > 0 \}$。

式\eqref{eq: coef1}和式\eqref{eq: coef2}中的系数如何得到呢?根据线性方程组理论,对$\Av \xv = \bv$的增广矩阵做初等行变换
\begin{align*}
    \begin{bmatrix}
        \Bv & \av_{m+1} & \cdots & \av_n & \bv
    \end{bmatrix} \longrightarrow
    \begin{bmatrix}
        \Iv_m & \Bv^{-1} \av_{m+1} & \cdots & \Bv^{-1} \av_n & \Bv^{-1} \bv
    \end{bmatrix}
\end{align*}
当基$\Bv$变成单位阵时,第$q$列和最后一列就是$\av_q$和$\bv$的线性表出系数。至此还剩两个问题:
\begin{enumerate}
    \item 如何确定入基向量$\av_q$;
    \item 如何确定当前解是否为最优解。
\end{enumerate}

下面考察基本可行解变化时目标函数值的变化,将标准型根据对$\Av$的分块重写为
\begin{align*}
    \max \quad & \cv_\Bv^\top \xv_\Bv + \cv_\Dv^\top \xv_\Dv \\
    \st  \quad & \Bv \xv_\Bv + \Dv \xv_\Dv = \bv             \\
               & \xv_\Bv, \xv_\Dv \ge \zerov
\end{align*}
\begin{itemize}
    \item 若$\xv_\Dv = \zerov$,则$\xv_\Bv = \Bv^{-1} \bv$,此时$\xv$就是关于基$\Bv$的基本可行解,对应的目标函数值为
          \begin{align*}
              \zhat = \cv_\Bv^\top \xv_\Bv = \cv_\Bv^\top \Bv^{-1} \bv
          \end{align*}
    \item 若$\xv_\Dv \ne \zerov$,则$\xv_\Bv = \Bv^{-1} \bv - \Bv^{-1} \Dv \xv_\Dv$,对应的目标函数值为
          \begin{align*}
              z = \cv_\Bv^\top \xv_\Bv + \cv_\Dv^\top \xv_\Dv = \cv_\Bv^\top (\Bv^{-1} \bv - \Bv^{-1} \Dv \xv_\Dv) + \cv_\Dv^\top \xv_\Dv = \cv_\Bv^\top \Bv^{-1} \bv - (\cv_\Bv^\top \Bv^{-1} \Dv - \cv_\Dv^\top) \xv_\Dv = \zhat - \rv_\Dv^\top \xv_\Dv
          \end{align*}
          其中$\rv_\Dv^\top = \cv_\Bv^\top \Bv^{-1} \Dv - \cv_\Dv^\top$称为\blue{检验数}。

          注意$\xv_\Dv \ge \zerov$,若$\rv_\Dv \ge \zerov$,则$z \ge \zhat$,即关于基$\Bv$的基本可行解就是最优解,这就回答了前面的问题2。若$\rv_\Dv$中某个分量为负,则将$\xv_\Dv$中对应的非基变量从$0$变为正数可使目标函数值变大,也即该非基变量对应的列入基,这就回答了前面的问题1。
\end{itemize}

基于此,构造单纯形表
\begin{align*}
    \begin{bmatrix}
        \Av & \bv \\ -\cv^\top & 0
    \end{bmatrix} =
    \begin{bmatrix}
        \Bv & \Dv & \bv \\ -\cv_\Bv^\top & -\cv_\Dv^\top & 0
    \end{bmatrix}
\end{align*}
先做初等行变换将基$\Bv$变成单位阵
\begin{align*}
    \begin{bmatrix}
        \Bv^{-1} & \zerov \\ \zerov^\top & 1
    \end{bmatrix}
    \begin{bmatrix}
        \Bv & \Dv & \bv \\ -\cv_\Bv^\top & -\cv_\Dv^\top & 0
    \end{bmatrix} =
    \begin{bmatrix}
        \Iv_m & \Bv^{-1} \Dv & \Bv^{-1} \bv \\ -\cv_\Bv^\top & -\cv_\Dv^\top & 0
    \end{bmatrix}
\end{align*}
再做初等行变换将最后一行基变量对应的$-\cv_\Bv^\top$变成$\zerov^\top$
\begin{align*}
    \begin{bmatrix}
        \Iv_m & \zerov \\ \cv_\Bv^\top & 1
    \end{bmatrix}
    \begin{bmatrix}
        \Iv_m & \Bv^{-1} \Dv & \Bv^{-1} \bv \\ -\cv_\Bv^\top & -\cv_\Dv^\top & 0
    \end{bmatrix} =
    \begin{bmatrix}
        \Iv_m & \Bv^{-1} \Dv & \Bv^{-1} \bv \\ \zerov^\top & \cv_\Bv^\top \Bv^{-1} \Dv - \cv_\Dv^\top & \cv_\Bv^\top \Bv^{-1} \bv
    \end{bmatrix}
\end{align*}
这张表里包含了一切我们需要的信息
\begin{itemize}
    \item $\Bv^{-1} \Dv$里的每列就是该列向量在当前基下的线性表示系数;
    \item $\Bv^{-1} \bv$是当前基对应的基本可行解中的基变量值;
    \item $\cv_\Bv^\top \Bv^{-1} \Dv - \cv_\Dv^\top$就是检验数,可以指示下一个入基向量和是否已达最优解;
    \item $\cv_\Bv^\top \Bv^{-1} \bv$就是当前基本可行解对应的目标函数值
\end{itemize}

\begin{example}
    用单纯形法再求例\ref{exam: illustration}中的线性规划,先转化为标准型:
    \begin{align*}
        \max \quad & 3 x_1 + 5 x_2                      \\
        \st  \quad & \begin{bmatrix}
                         1 & 5 & 1 &       \\
                         2 & 1 &   & 1 &   \\
                         1 & 1 &   &   & 1
                     \end{bmatrix} \xv = \begin{bmatrix}
                                             40 \\ 20 \\ 12
                                         \end{bmatrix} \\
                   & \xv \ge 0
    \end{align*}

    初始单纯形表为
    \begin{align*}
        \begin{array}{c|ccccc:c}
                & x_1 & x_2 & x_3 & x_4 & x_5      \\ \hline
            x_3 & 1   & 5   & 1   &     &     & 40 \\
            x_4 & 2   & 1   &     & 1   &     & 20 \\
            x_5 & 1   & 1   &     &     & 1   & 12 \\ \hdashline
                & -3  & -5  &     &     &     & 0
        \end{array}
    \end{align*}
    此时$x_3$、$x_4$、$x_5$是基变量,基本可行解为
    \begin{align*}
        \left[ \begin{array}{ccccc:c}
                       x_1 & x_2 & x_3 & x_4 & x_5 & o \\
                       0   & 0   & 40  & 20  & 12  & 0
                   \end{array} \right]
    \end{align*}
    对应$\Rbb^2$中可行域的极点$[0,0]$,由于检验数还有负值,因此还不是最优解。

    取检验数绝对值最大的负数对应的列入基,即$\av_2$入基。注意
    \begin{align*}
        \bv = 40 \av_3 + 20 \av_4 + 12 \av_5, \quad \av_2 = 5 \av_3 + 1 \av_4 + 1 \av_5
    \end{align*}
    计算$\argmin\{\nicefrac{40}{5}, \nicefrac{20}{1}, \nicefrac{12}{1}\}$可知$\av_3$出基。做初等行变换
    \begin{align*}
        \begin{array}{c|ccccc:c}
                & x_1 & x_2 & x_3 & x_4 & x_5      \\ \hline
            x_2 & 0.2 & 1   & 0.2 &     &     & 8  \\
            x_4 & 2   & 1   &     & 1   &     & 20 \\
            x_5 & 1   & 1   &     &     & 1   & 12 \\ \hdashline
                & -3  & -5  &     &     &     & 0
        \end{array} \Longrightarrow
        \begin{array}{c|ccccc:c}
                & x_1 & x_2 & x_3  & x_4 & x_5      \\ \hline
            x_2 & 0.2 & 1   & 0.2  &     &     & 8  \\
            x_4 & 1.8 &     & -0.2 & 1   &     & 12 \\
            x_5 & 0.8 &     & -0.2 &     & 1   & 4  \\ \hdashline
                & -2  &     & 1    &     &     & 40
        \end{array}
    \end{align*}
    此时$x_2$、$x_4$、$x_5$是基变量,基本可行解为
    \begin{align*}
        \left[ \begin{array}{ccccc:c}
                       x_1 & x_2 & x_3 & x_4 & x_5 & o  \\
                       0   & 8   & 0   & 12  & 4   & 40
                   \end{array} \right]
    \end{align*}
    对应$\Rbb^2$中可行域的极点$[0,8]$,由于检验数还有负值,因此还不是最优解。

    根据检验数$\av_1$入基,计算$\argmin\{\nicefrac{8}{0.2}, \nicefrac{12}{1.8}, \nicefrac{4}{0.8}\}$可知$\av_5$出基。做初等行变换
    \begin{align*}
        \begin{array}{c|ccccc:c}
                & x_1 & x_2 & x_3   & x_4 & x_5       \\ \hline
            x_2 & 0.2 & 1   & 0.2   &     &      & 8  \\
            x_4 & 1.8 &     & -0.2  & 1   &      & 12 \\
            x_1 & 1   &     & -0.25 &     & 1.25 & 5  \\ \hdashline
                & -2  &     & 1     &     &      & 40
        \end{array} \Longrightarrow
        \begin{array}{c|ccccc:c}
                & x_1 & x_2 & x_3   & x_4 & x_5        \\ \hline
            x_2 &     & 1   & 0.25  &     & -0.25 & 7  \\
            x_4 &     &     & 0.25  & 1   & -2.25 & 3  \\
            x_1 & 1   &     & -0.25 &     & 1.25  & 5  \\ \hdashline
                &     &     & 0.5   &     & 2.5   & 50
        \end{array}
    \end{align*}
    此时$x_1$、$x_2$、$x_4$是基变量,基本可行解为
    \begin{align*}
        \left[ \begin{array}{ccccc:c}
                       x_1 & x_2 & x_3 & x_4 & x_5 & o  \\
                       5   & 7   & 0   & 3   & 0   & 50
                   \end{array} \right]
    \end{align*}
    对应$\Rbb^2$中可行域的极点$[5,7]$,由于检验数均非负,已达最优解。
\end{example}

\begin{example}
    用单纯形法求例\ref{exam: bag}中的分数背包问题,先转化为标准型:
    \begin{align*}
        \max \quad & 40 x_1 + 42 x_2 + 25 x_3 + 12 x_4 \\
        \st \quad  & \begin{bmatrix}
                         4 & 7 & 5 & 3 & 1                 \\
                         1 &   &   &   &   & 1             \\
                           & 1 &   &   &   &   & 1         \\
                           &   & 1 &   &   &   &   & 1     \\
                           &   &   & 1 &   &   &   &   & 1 \\
                     \end{bmatrix} \xv =
        \begin{bmatrix}
            10 \\ 1 \\ 1 \\ 1 \\ 1
        \end{bmatrix}                          \\
                   & \xv \ge 0
    \end{align*}

    初始单纯形表为
    \begin{align*}
        \begin{array}{c|ccccccccc:c}
                & x_1 & x_2 & x_3 & x_4 & x_5 & x_6 & x_7 & x_8 & x_9      \\ \hline
            x_5 & 4   & 7   & 5   & 3   & 1   &     &     &     &     & 10 \\
            x_6 & 1   &     &     &     &     & 1   &     &     &     & 1  \\
            x_7 &     & 1   &     &     &     &     & 1   &     &     & 1  \\
            x_8 &     &     & 1   &     &     &     &     & 1   &     & 1  \\
            x_9 &     &     &     & 1   &     &     &     &     & 1   & 1  \\ \hdashline
                & -40 & -42 & -25 & -12 &     &     &     &     &     & 0
        \end{array}
    \end{align*}
    此时$x_5$、$x_6$、$x_7$、$x_8$、$x_9$是基变量,基本可行解为
    \begin{align*}
        \left[ \begin{array}{ccccccccc:c}
                       x_1 & x_2 & x_3 & x_4 & x_5 & x_6 & x_7 & x_8 & x_9 & o \\
                       0   & 0   & 0   & 0   & 10  & 1   & 1   & 1   & 1   & 0
                   \end{array} \right]
    \end{align*}

    根据检验数$\av_2$入基,计算$\argmin\{\nicefrac{10}{7}, \nicefrac{1}{1}\}$可知$\av_7$出基。做初等行变换
    \begin{align*}
        \begin{array}{c|ccccccccc:c}
                & x_1 & x_2 & x_3 & x_4 & x_5 & x_6 & x_7 & x_8 & x_9      \\ \hline
            x_5 & 4   &     & 5   & 3   & 1   &     & -7  &     &     & 3  \\
            x_6 & 1   &     &     &     &     & 1   &     &     &     & 1  \\
            x_2 &     & 1   &     &     &     &     & 1   &     &     & 1  \\
            x_8 &     &     & 1   &     &     &     &     & 1   &     & 1  \\
            x_9 &     &     &     & 1   &     &     &     &     & 1   & 1  \\ \hdashline
                & -40 &     & -25 & -12 &     &     & 42  &     &     & 42
        \end{array}
    \end{align*}
    此时$x_2$、$x_5$、$x_6$、$x_8$、$x_9$是基变量,基本可行解为
    \begin{align*}
        \left[ \begin{array}{ccccccccc:c}
                       x_1 & x_2 & x_3 & x_4 & x_5 & x_6 & x_7 & x_8 & x_9 & o  \\
                       0   & 1   & 0   & 0   & 3   & 1   & 0   & 1   & 1   & 42
                   \end{array} \right]
    \end{align*}

    根据检验数$\av_1$入基,计算$\argmin\{\nicefrac{3}{4}, \nicefrac{1}{1}\}$可知$\av_5$出基。做初等行变换
    \begin{align*}\arraycolsep=3pt
        \begin{array}{c|ccccccccc:c}
                & x_1 & x_2 & x_3             & x_4             & x_5             & x_6 & x_7              & x_8 & x_9                   \\ \hline
            x_1 & 1   &     & \nicefrac{5}{4} & \nicefrac{3}{4} & \nicefrac{1}{4} &     & \nicefrac{-7}{4} &     &     & \nicefrac{3}{4} \\
            x_6 & 1   &     &                 &                 &                 & 1   &                  &     &     & 1               \\
            x_2 &     & 1   &                 &                 &                 &     & 1                &     &     & 1               \\
            x_8 &     &     & 1               &                 &                 &     &                  & 1   &     & 1               \\
            x_9 &     &     &                 & 1               &                 &     &                  &     & 1   & 1               \\ \hdashline
                & -40 &     & -25             & -12             &                 &     & 42               &     &     & 42
        \end{array} \Longrightarrow
        \begin{array}{c|ccccccccc:c}
                & x_1 & x_2 & x_3              & x_4              & x_5              & x_6 & x_7              & x_8 & x_9                   \\ \hline
            x_1 & 1   &     & \nicefrac{5}{4}  & \nicefrac{3}{4}  & \nicefrac{1}{4}  &     & \nicefrac{-7}{4} &     &     & \nicefrac{3}{4} \\
            x_6 &     &     & \nicefrac{-5}{4} & \nicefrac{-3}{4} & \nicefrac{-1}{4} & 1   & \nicefrac{7}{4}  &     &     & \nicefrac{1}{4} \\
            x_2 &     & 1   &                  &                  &                  &     & 1                &     &     & 1               \\
            x_8 &     &     & 1                &                  &                  &     &                  & 1   &     & 1               \\
            x_9 &     &     &                  & 1                &                  &     &                  &     & 1   & 1               \\ \hdashline
                &     &     & 25               & 18               & 10               & 40  & -28              &     &     & 72
        \end{array}
    \end{align*}
    此时$x_1$、$x_2$、$x_6$、$x_8$、$x_9$是基变量,基本可行解为
    \begin{align*}
        \left[ \begin{array}{ccccccccc:c}
                       x_1             & x_2 & x_3 & x_4 & x_5 & x_6             & x_7 & x_8 & x_9 & o  \\
                       \nicefrac{3}{4} & 1   & 0   & 0   & 0   & \nicefrac{1}{4} & 0   & 1   & 1   & 72
                   \end{array} \right]
    \end{align*}

    根据检验数$\av_7$入基,计算$\argmin\{\nicefrac{1}{7}, \nicefrac{1}{1}\}$可知$\av_6$出基。做初等行变换
    \begin{align*}\arraycolsep=3pt
        \begin{array}{c|ccccccccc:c}
                & x_1 & x_2 & x_3              & x_4              & x_5              & x_6 & x_7              & x_8 & x_9                   \\ \hline
            x_1 & 1   &     & \nicefrac{5}{4}  & \nicefrac{3}{4}  & \nicefrac{1}{4}  &     & \nicefrac{-7}{4} &     &     & \nicefrac{3}{4} \\
            x_7 &     &     & \nicefrac{-5}{7} & \nicefrac{-3}{7} & \nicefrac{-1}{7} & 1   & 1                &     &     & \nicefrac{1}{7} \\
            x_2 &     & 1   &                  &                  &                  &     & 1                &     &     & 1               \\
            x_8 &     &     & 1                &                  &                  &     &                  & 1   &     & 1               \\
            x_9 &     &     &                  & 1                &                  &     &                  &     & 1   & 1               \\ \hdashline
                &     &     & 25               & 18               & 10               & 40  & -28              &     &     & 72
        \end{array} \Longrightarrow
        \begin{array}{c|ccccccccc:c}
                & x_1 & x_2 & x_3              & x_4              & x_5              & x_6             & x_7 & x_8 & x_9                   \\ \hline
            x_1 & 1   &     &                  &                  &                  & \nicefrac{7}{4} &     &     &     & 1               \\
            x_7 &     &     & \nicefrac{-5}{7} & \nicefrac{-3}{7} & \nicefrac{-1}{7} & 1               & 1   &     &     & \nicefrac{1}{7} \\
            x_2 &     & 1   & \nicefrac{5}{7}  & \nicefrac{3}{7}  & \nicefrac{1}{7}  & -1              &     &     &     & \nicefrac{6}{7} \\
            x_8 &     &     & 1                &                  &                  &                 &     & 1   &     & 1               \\
            x_9 &     &     &                  & 1                &                  &                 &     &     & 1   & 1               \\ \hdashline
                &     &     & 5                & 6                & 6                & 68              &     &     &     & 76
        \end{array}
    \end{align*}
    此时$x_1$、$x_2$、$x_7$、$x_8$、$x_9$是基变量,基本可行解为
    \begin{align*}
        \left[ \begin{array}{ccccccccc:c}
                       x_1 & x_2             & x_3 & x_4 & x_5 & x_6 & x_7             & x_8 & x_9 & o  \\
                       1   & \nicefrac{6}{7} & 0   & 0   & 0   & 0   & \nicefrac{1}{7} & 1   & 1   & 76
                   \end{array} \right]
    \end{align*}
    由于检验数均非负,已达最优解。

\end{example}

\begin{remark}
    分数背包问题也可采用贪心法来做。
\end{remark}

\begin{example}
    用单纯形法求例\ref{exam: flow}中的最大流问题,先转化为标准型:
    \begin{align*}
        \max \quad & x_1 + x_2                 \\
        \st \quad  & x_1 + y_1 = 16            \\
                   & x_2 + y_2 =  13           \\
                   & x_3 + y_3 =  4            \\
                   & x_4 + y_4 =  12           \\
                   & x_5 + y_5 =  9            \\
                   & x_6 + y_6 =  14           \\
                   & x_7 + y_7 =  7            \\
                   & x_8 + y_8 =  20           \\
                   & x_9 + y_9 =  4            \\
                   & x_1 + x_3 - x_4 = 0       \\
                   & x_2 + x_5 - x_3 - x_6 = 0 \\
                   & x_4 + x_7 - x_5 - x_8 = 0 \\
                   & x_6 - x_7 - x_9 = 0       \\
                   & \xv, \yv \ge \zerov
    \end{align*}
    共有$18$个变量、$13$个等式约束,因此基变量有$13$个,非基变量有$5$个。初始不妨取$x_1$、$x_2$、$x_4$、$x_5$、$x_7$为非基变量,将基变量由$x_1$、$x_2$、$x_4$、$x_5$、$x_7$表出:
    \begin{align*}
        \begin{array}{rclcl}
            x_3 = -x_1 + x_4      & \Rightarrow & x_1 + x_3 - x_4 = 0                    & \Rightarrow & -x_1 + x_4 + y_3 = 4                  \\
            x_8 = x_4 - x_5 + x_7 & \Rightarrow & -x_4 + x_5 - x_7 + x_8 = 0             & \Rightarrow & x_4 - x_5 + x_7 + y_8 = 20            \\
            x_6 = x_2 + x_5 - x_3 & \Rightarrow & -x_1 - x_2 + x_4 - x_5 + x_6 = 0       & \Rightarrow & x_1 + x_2 - x_4 + x_5 + y_6 = 14      \\
            x_9 = x_6 - x_7       & \Rightarrow & -x_1 - x_2 + x_4 - x_5 + x_7 + x_9 = 0 & \Rightarrow & x_1 + x_2 - x_4 + x_5 - x_7 + y_9 = 4 \\
        \end{array}
    \end{align*}

    初始单纯形表为
    \begin{align*}
        \begin{array}{c|cccccccccccccccccc:c}
                & x_1 & x_2 & x_3 & x_4 & x_5 & x_6 & x_7 & x_8 & x_9 & y_1 & y_2 & y_3 & y_4 & y_5 & y_6 & y_7 & y_8 & y_9      \\ \hline
            x_3 & 1   &     & 1   & -1  &     &     &     &     &     &     &     &     &     &     &     &     &     &     & 0  \\
            x_6 & -1  & -1  &     & 1   & -1  & 1   &     &     &     &     &     &     &     &     &     &     &     &     & 0  \\
            x_8 &     &     &     & -1  & 1   &     & -1  & 1   &     &     &     &     &     &     &     &     &     &     & 0  \\
            x_9 & -1  & -1  &     & 1   & -1  &     & 1   &     & 1   &     &     &     &     &     &     &     &     &     & 0  \\
            y_1 & 1   &     &     &     &     &     &     &     &     & 1   &     &     &     &     &     &     &     &     & 16 \\
            y_2 &     & 1   &     &     &     &     &     &     &     &     & 1   &     &     &     &     &     &     &     & 13 \\
            y_3 & -1  &     &     & 1   &     &     &     &     &     &     &     & 1   &     &     &     &     &     &     & 4  \\
            y_4 &     &     &     & 1   &     &     &     &     &     &     &     &     & 1   &     &     &     &     &     & 12 \\
            y_5 &     &     &     &     & 1   &     &     &     &     &     &     &     &     & 1   &     &     &     &     & 9  \\
            y_6 & 1   & 1   &     & -1  & 1   &     &     &     &     &     &     &     &     &     & 1   &     &     &     & 14 \\
            y_7 &     &     &     &     &     &     & 1   &     &     &     &     &     &     &     &     & 1   &     &     & 7  \\
            y_8 &     &     &     & 1   & -1  &     & 1   &     &     &     &     &     &     &     &     &     & 1   &     & 20 \\
            y_9 & 1   & 1   &     & -1  & 1   &     & -1  &     &     &     &     &     &     &     &     &     &     & 1   & 4  \\ \hdashline
                & -1  & -1  &     &     &     &     &     &     &     &     &     &     &     &     &     &     &     &     & 0
        \end{array}
    \end{align*}
    基本可行解为
    \begin{align*}
        \left[
            \begin{array}{cccccccccccccccccc:c}
                x_1 & x_2 & x_3 & x_4 & x_5 & x_6 & x_7 & x_8 & x_9 & y_1 & y_2 & y_3 & y_4 & y_5 & y_6 & y_7 & y_8 & y_9 & o \\
                0   & 0   & 0   & 0   & 0   & 0   & 0   & 0   & 0   & 16  & 13  & 4   & 12  & 9   & 14  & 7   & 20  & 4   & 0
            \end{array} \right]
    \end{align*}
    对应的流网络为
    \begin{figure}[h]
        \centering
        \begin{tikzpicture}

            \pgfmathsetmacro{\l}{3.5};

            \node [point] (s) at (0,0) {$\sv$};
            \path (s) ++(30:\l)  node[point] (v1) {$\vv_1$};
            \path (s) ++(330:\l)  node[point] (v2) {$\vv_2$};
            \path (v1) ++(\l,0)  node[point] (v3) {$\vv_3$};
            \path (v2) ++(\l,0)  node[point] (v4) {$\vv_4$};
            \path (v4) ++(30:\l)  node[point] (t) {$\tv$};

            \draw [arrow] (s) -- (v1) node [above=0pt, sloped, pos=0.5] {$16$};
            \draw [arrow] (s) -- (v2) node [below=0pt, sloped, pos=0.4] {$13$};
            \draw [arrow] (v2) -- (v1) node [above=0pt, sloped, pos=0.5] {$4$};
            \draw [arrow] (v1) -- (v3) node [above=0pt, pos=0.5] {$12$};
            \draw [arrow] (v2) -- (v4) node [below=0pt, pos=0.5] {$14$};
            \draw [arrow] (v4) -- (v3) node [above=0pt, sloped, pos=0.5] {$7$};
            \draw [arrow] (v3) -- (v2) node [above=0pt, sloped, pos=0.5] {$9$};
            \draw [arrow] (v3) -- (t) node [above=0pt, sloped, pos=0.5] {$20$};
            \draw [arrow] (v4) -- (t) node [below=0pt, sloped, pos=0.5] {$4$};

        \end{tikzpicture}
    \end{figure}

    \newpage

    $\av_1$、$\av_2$的检验数均为$-1$,不妨让$\av_2$入基,计算$\argmin\{\nicefrac{13}{1}, \nicefrac{14}{1}, \nicefrac{4}{1}\}$可知$\av_{18}$出基。做初等行变换
    \begin{align*}
        \begin{array}{c|cccccccccccccccccc:c}
                & x_1 & x_2 & x_3 & x_4 & x_5 & x_6 & x_7 & x_8 & x_9 & y_1 & y_2 & y_3 & y_4 & y_5 & y_6 & y_7 & y_8 & y_9      \\ \hline
            x_3 & 1   &     & 1   & -1  &     &     &     &     &     &     &     &     &     &     &     &     &     &     & 0  \\
            x_6 &     &     &     &     &     & 1   & -1  &     &     &     &     &     &     &     &     &     &     & -1  & 4  \\
            x_8 &     &     &     & -1  & 1   &     & -1  & 1   &     &     &     &     &     &     &     &     &     &     & 0  \\
            x_9 &     &     &     &     &     &     &     &     & 1   &     &     &     &     &     &     &     &     & 1   & 4  \\
            y_1 & 1   &     &     &     &     &     &     &     &     & 1   &     &     &     &     &     &     &     &     & 16 \\
            y_2 & -1  &     &     & 1   & -1  &     & 1   &     &     &     & 1   &     &     &     &     &     &     & -1  & 9  \\
            y_3 & -1  &     &     & 1   &     &     &     &     &     &     &     & 1   &     &     &     &     &     &     & 4  \\
            y_4 &     &     &     & 1   &     &     &     &     &     &     &     &     & 1   &     &     &     &     &     & 12 \\
            y_5 &     &     &     &     & 1   &     &     &     &     &     &     &     &     & 1   &     &     &     &     & 9  \\
            y_6 &     &     &     &     &     &     & 1   &     &     &     &     &     &     &     & 1   &     &     & -1  & 10 \\
            y_7 &     &     &     &     &     &     & 1   &     &     &     &     &     &     &     &     & 1   &     &     & 7  \\
            y_8 &     &     &     & 1   & -1  &     & 1   &     &     &     &     &     &     &     &     &     & 1   &     & 20 \\
            x_2 & 1   & 1   &     & -1  & 1   &     & -1  &     &     &     &     &     &     &     &     &     &     & 1   & 4  \\ \hdashline
                &     &     &     & -1  & 1   &     & -1  &     &     &     &     &     &     &     &     &     &     & 1   & 4
        \end{array}
    \end{align*}
    当前基本可行解为
    \begin{align*}
        \left[
            \begin{array}{cccccccccccccccccc:c}
                x_1 & x_2 & x_3 & x_4 & x_5 & x_6 & x_7 & x_8 & x_9 & y_1 & y_2 & y_3 & y_4 & y_5 & y_6 & y_7 & y_8 & y_9 & o \\
                0   & 4   & 0   & 0   & 0   & 4   & 0   & 0   & 4   & 16  & 9   & 4   & 12  & 9   & 10  & 7   & 20  & 0   & 4
            \end{array} \right]
    \end{align*}
    对应的流网络为
    \begin{figure}[h]
        \centering
        \begin{tikzpicture}

            \pgfmathsetmacro{\l}{3.5};

            \node [point] (s) at (0,0) {$\sv$};
            \path (s) ++(30:\l)  node[point] (v1) {$\vv_1$};
            \path (s) ++(330:\l)  node[point] (v2) {$\vv_2$};
            \path (v1) ++(\l,0)  node[point] (v3) {$\vv_3$};
            \path (v2) ++(\l,0)  node[point] (v4) {$\vv_4$};
            \path (v4) ++(30:\l)  node[point] (t) {$\tv$};

            \draw [arrow] (s) -- (v1) node [above=0pt, sloped, pos=0.5] {$16$};
            \draw [arrow] (s) -- (v2) node [below=0pt, sloped, pos=0.4] {$4/13$};
            \draw [arrow] (v2) -- (v1) node [above=0pt, sloped, pos=0.5] {$4$};
            \draw [arrow] (v1) -- (v3) node [above=0pt, pos=0.5] {$12$};
            \draw [arrow] (v2) -- (v4) node [below=0pt, pos=0.5] {$4/14$};
            \draw [arrow] (v4) -- (v3) node [above=0pt, sloped, pos=0.5] {$7$};
            \draw [arrow] (v3) -- (v2) node [above=0pt, sloped, pos=0.5] {$9$};
            \draw [arrow] (v3) -- (t) node [above=0pt, sloped, pos=0.5] {$20$};
            \draw [arrow] (v4) -- (t) node [below=0pt, sloped, pos=0.5] {$4/4$};

        \end{tikzpicture}
    \end{figure}

    \newpage

    $\av_4$、$\av_7$的检验数均为$-1$,不妨让$\av_7$入基,计算$\argmin\{\nicefrac{9}{1}, \nicefrac{10}{1}, \nicefrac{7}{1}, \nicefrac{20}{1}\}$可知$\av_{16}$出基。做初等行变换
    \begin{align*}
        \begin{array}{c|cccccccccccccccccc:c}
                & x_1 & x_2 & x_3 & x_4 & x_5 & x_6 & x_7 & x_8 & x_9 & y_1 & y_2 & y_3 & y_4 & y_5 & y_6 & y_7 & y_8 & y_9      \\ \hline
            x_3 & 1   &     & 1   & -1  &     &     &     &     &     &     &     &     &     &     &     &     &     &     & 0  \\
            x_6 &     &     &     &     &     & 1   &     &     &     &     &     &     &     &     &     & 1   &     & -1  & 11 \\
            x_8 &     &     &     & -1  & 1   &     &     & 1   &     &     &     &     &     &     &     & 1   &     &     & 7  \\
            x_9 &     &     &     &     &     &     &     &     & 1   &     &     &     &     &     &     &     &     & 1   & 4  \\
            y_1 & 1   &     &     &     &     &     &     &     &     & 1   &     &     &     &     &     &     &     &     & 16 \\
            y_2 & -1  &     &     & 1   & -1  &     &     &     &     &     & 1   &     &     &     &     & -1  &     & -1  & 2  \\
            y_3 & -1  &     &     & 1   &     &     &     &     &     &     &     & 1   &     &     &     &     &     &     & 4  \\
            y_4 &     &     &     & 1   &     &     &     &     &     &     &     &     & 1   &     &     &     &     &     & 12 \\
            y_5 &     &     &     &     & 1   &     &     &     &     &     &     &     &     & 1   &     &     &     &     & 9  \\
            y_6 &     &     &     &     &     &     &     &     &     &     &     &     &     &     & 1   & -1  &     & -1  & 3  \\
            x_7 &     &     &     &     &     &     & 1   &     &     &     &     &     &     &     &     & 1   &     &     & 7  \\
            y_8 &     &     &     & 1   & -1  &     &     &     &     &     &     &     &     &     &     & -1  & 1   &     & 13 \\
            x_2 & 1   & 1   &     & -1  & 1   &     &     &     &     &     &     &     &     &     &     & 1   &     & 1   & 11 \\ \hdashline
                &     &     &     & -1  & 1   &     &     &     &     &     &     &     &     &     &     & 1   &     & 1   & 11
        \end{array}
    \end{align*}
    当前基本可行解为
    \begin{align*}
        \left[
            \begin{array}{cccccccccccccccccc:c}
                x_1 & x_2 & x_3 & x_4 & x_5 & x_6 & x_7 & x_8 & x_9 & y_1 & y_2 & y_3 & y_4 & y_5 & y_6 & y_7 & y_8 & y_9 & o  \\
                0   & 11  & 0   & 0   & 0   & 11  & 7   & 7   & 4   & 16  & 2   & 4   & 12  & 9   & 3   & 0   & 13  & 0   & 11
            \end{array} \right]
    \end{align*}
    对应的流网络为
    \begin{figure}[h]
        \centering
        \begin{tikzpicture}

            \pgfmathsetmacro{\l}{3.5};

            \node [point] (s) at (0,0) {$\sv$};
            \path (s) ++(30:\l)  node[point] (v1) {$\vv_1$};
            \path (s) ++(330:\l)  node[point] (v2) {$\vv_2$};
            \path (v1) ++(\l,0)  node[point] (v3) {$\vv_3$};
            \path (v2) ++(\l,0)  node[point] (v4) {$\vv_4$};
            \path (v4) ++(30:\l)  node[point] (t) {$\tv$};

            \draw [arrow] (s) -- (v1) node [above=0pt, sloped, pos=0.5] {$16$};
            \draw [arrow] (s) -- (v2) node [below=0pt, sloped, pos=0.4] {$11/13$};
            \draw [arrow] (v2) -- (v1) node [above=0pt, sloped, pos=0.5] {$4$};
            \draw [arrow] (v1) -- (v3) node [above=0pt, pos=0.5] {$12$};
            \draw [arrow] (v2) -- (v4) node [below=0pt, pos=0.5] {$11/14$};
            \draw [arrow] (v4) -- (v3) node [above=0pt, sloped, pos=0.5] {$7/7$};
            \draw [arrow] (v3) -- (v2) node [above=0pt, sloped, pos=0.5] {$9$};
            \draw [arrow] (v3) -- (t) node [above=0pt, sloped, pos=0.5] {$7/20$};
            \draw [arrow] (v4) -- (t) node [below=0pt, sloped, pos=0.5] {$4/4$};

        \end{tikzpicture}
    \end{figure}

    \newpage

    根据检验数$\av_4$入基,计算$\argmin\{\nicefrac{2}{1}, \nicefrac{4}{1}, \nicefrac{12}{1}, \nicefrac{13}{1}\}$可知$\av_{11}$出基。做初等行变换
    \begin{align*}
        \begin{array}{c|cccccccccccccccccc:c}
                & x_1 & x_2 & x_3 & x_4 & x_5 & x_6 & x_7 & x_8 & x_9 & y_1 & y_2 & y_3 & y_4 & y_5 & y_6 & y_7 & y_8 & y_9      \\ \hline
            x_3 &     &     & 1   &     & -1  &     &     &     &     &     & 1   &     &     &     &     & -1  &     & -1  & 2  \\
            x_6 &     &     &     &     &     & 1   &     &     &     &     &     &     &     &     &     & 1   &     & -1  & 11 \\
            x_8 & -1  &     &     &     &     &     &     & 1   &     &     & 1   &     &     &     &     &     &     & -1  & 9  \\
            x_9 &     &     &     &     &     &     &     &     & 1   &     &     &     &     &     &     &     &     & 1   & 4  \\
            y_1 & 1   &     &     &     &     &     &     &     &     & 1   &     &     &     &     &     &     &     &     & 16 \\
            x_4 & -1  &     &     & 1   & -1  &     &     &     &     &     & 1   &     &     &     &     & -1  &     & -1  & 2  \\
            y_3 &     &     &     &     & 1   &     &     &     &     &     & -1  & 1   &     &     &     & 1   &     & 1   & 2  \\
            y_4 & 1   &     &     &     & 1   &     &     &     &     &     & -1  &     & 1   &     &     & 1   &     & 1   & 10 \\
            y_5 &     &     &     &     & 1   &     &     &     &     &     &     &     &     & 1   &     &     &     &     & 9  \\
            y_6 &     &     &     &     &     &     &     &     &     &     &     &     &     &     & 1   & -1  &     & -1  & 3  \\
            x_7 &     &     &     &     &     &     & 1   &     &     &     &     &     &     &     &     & 1   &     &     & 7  \\
            y_8 & 1   &     &     &     &     &     &     &     &     &     & -1  &     &     &     &     &     & 1   & 1   & 11 \\
            x_2 &     & 1   &     &     &     &     &     &     &     &     & 1   &     &     &     &     &     &     &     & 13 \\ \hdashline
                & -1  &     &     &     &     &     &     &     &     &     & 1   &     &     &     &     &     &     &     & 13
        \end{array}
    \end{align*}
    当前基本可行解为
    \begin{align*}
        \left[
            \begin{array}{cccccccccccccccccc:c}
                x_1 & x_2 & x_3 & x_4 & x_5 & x_6 & x_7 & x_8 & x_9 & y_1 & y_2 & y_3 & y_4 & y_5 & y_6 & y_7 & y_8 & y_9 & o  \\
                0   & 13  & 2   & 2   & 0   & 11  & 7   & 9   & 4   & 16  & 0   & 2   & 10  & 9   & 3   & 0   & 11  & 0   & 13
            \end{array} \right]
    \end{align*}
    对应的流网络为
    \begin{figure}[h]
        \centering
        \begin{tikzpicture}

            \pgfmathsetmacro{\l}{3.5};

            \node [point] (s) at (0,0) {$\sv$};
            \path (s) ++(30:\l)  node[point] (v1) {$\vv_1$};
            \path (s) ++(330:\l)  node[point] (v2) {$\vv_2$};
            \path (v1) ++(\l,0)  node[point] (v3) {$\vv_3$};
            \path (v2) ++(\l,0)  node[point] (v4) {$\vv_4$};
            \path (v4) ++(30:\l)  node[point] (t) {$\tv$};

            \draw [arrow] (s) -- (v1) node [above=0pt, sloped, pos=0.5] {$16$};
            \draw [arrow] (s) -- (v2) node [below=0pt, sloped, pos=0.4] {$13/13$};
            \draw [arrow] (v2) -- (v1) node [above=0pt, sloped, pos=0.5] {$2/4$};
            \draw [arrow] (v1) -- (v3) node [above=0pt, pos=0.5] {$2/12$};
            \draw [arrow] (v2) -- (v4) node [below=0pt, pos=0.5] {$11/14$};
            \draw [arrow] (v4) -- (v3) node [above=0pt, sloped, pos=0.5] {$7/7$};
            \draw [arrow] (v3) -- (v2) node [above=0pt, sloped, pos=0.5] {$9$};
            \draw [arrow] (v3) -- (t) node [above=0pt, sloped, pos=0.5] {$9/20$};
            \draw [arrow] (v4) -- (t) node [below=0pt, sloped, pos=0.5] {$4/4$};

        \end{tikzpicture}
    \end{figure}

    \newpage

    根据检验数$\av_1$入基,计算$\argmin\{\nicefrac{16}{1}, \nicefrac{10}{1}, \nicefrac{11}{1}\}$可知$\av_{13}$出基。做初等行变换
    \begin{align*}
        \begin{array}{c|cccccccccccccccccc:c}
                & x_1 & x_2 & x_3 & x_4 & x_5 & x_6 & x_7 & x_8 & x_9 & y_1 & y_2 & y_3 & y_4 & y_5 & y_6 & y_7 & y_8 & y_9      \\ \hline
            x_3 &     &     & 1   &     & -1  &     &     &     &     &     & 1   &     &     &     &     & -1  &     & -1  & 2  \\
            x_6 &     &     &     &     &     & 1   &     &     &     &     &     &     &     &     &     & 1   &     & -1  & 11 \\
            x_8 &     &     &     &     & 1   &     &     & 1   &     &     &     &     & 1   &     &     & 1   &     &     & 19 \\
            x_9 &     &     &     &     &     &     &     &     & 1   &     &     &     &     &     &     &     &     & 1   & 4  \\
            y_1 &     &     &     &     & -1  &     &     &     &     & 1   & 1   &     & -1  &     &     & -1  &     & -1  & 6  \\
            x_4 &     &     &     & 1   &     &     &     &     &     &     &     &     & 1   &     &     &     &     &     & 12 \\
            y_3 &     &     &     &     & 1   &     &     &     &     &     & -1  & 1   &     &     &     & 1   &     & 1   & 2  \\
            x_1 & 1   &     &     &     & 1   &     &     &     &     &     & -1  &     & 1   &     &     & 1   &     & 1   & 10 \\
            y_5 &     &     &     &     & 1   &     &     &     &     &     &     &     &     & 1   &     &     &     &     & 9  \\
            y_6 &     &     &     &     &     &     &     &     &     &     &     &     &     &     & 1   & -1  &     & -1  & 3  \\
            x_7 &     &     &     &     &     &     & 1   &     &     &     &     &     &     &     &     & 1   &     &     & 7  \\
            y_8 &     &     &     &     & -1  &     &     &     &     &     &     &     & -1  &     &     & -1  & 1   &     & 1  \\
            x_2 &     & 1   &     &     &     &     &     &     &     &     & 1   &     &     &     &     &     &     &     & 13 \\ \hdashline
                &     &     &     &     & 1   &     &     &     &     &     &     &     & 1   &     &     & 1   &     & 1   & 23
        \end{array}
    \end{align*}
    当前基本可行解为
    \begin{align*}
        \left[
            \begin{array}{cccccccccccccccccc:c}
                x_1 & x_2 & x_3 & x_4 & x_5 & x_6 & x_7 & x_8 & x_9 & y_1 & y_2 & y_3 & y_4 & y_5 & y_6 & y_7 & y_8 & y_9 & o  \\
                10  & 13  & 2   & 12  & 0   & 11  & 7   & 19  & 4   & 6   & 0   & 2   & 0   & 9   & 3   & 0   & 1   & 0   & 23
            \end{array} \right]
    \end{align*}
    对应的流网络为
    \begin{figure}[h]
        \centering
        \begin{tikzpicture}

            \pgfmathsetmacro{\l}{3.5};

            \node [point] (s) at (0,0) {$\sv$};
            \path (s) ++(30:\l)  node[point] (v1) {$\vv_1$};
            \path (s) ++(330:\l)  node[point] (v2) {$\vv_2$};
            \path (v1) ++(\l,0)  node[point] (v3) {$\vv_3$};
            \path (v2) ++(\l,0)  node[point] (v4) {$\vv_4$};
            \path (v4) ++(30:\l)  node[point] (t) {$\tv$};

            \draw [arrow] (s) -- (v1) node [above=0pt, sloped, pos=0.5] {$10/16$};
            \draw [arrow] (s) -- (v2) node [below=0pt, sloped, pos=0.4] {$13/13$};
            \draw [arrow] (v2) -- (v1) node [above=0pt, sloped, pos=0.5] {$2/4$};
            \draw [arrow] (v1) -- (v3) node [above=0pt, pos=0.5] {$12/12$};
            \draw [arrow] (v2) -- (v4) node [below=0pt, pos=0.5] {$11/14$};
            \draw [arrow] (v4) -- (v3) node [above=0pt, sloped, pos=0.5] {$7/7$};
            \draw [arrow] (v3) -- (v2) node [above=0pt, sloped, pos=0.5] {$9$};
            \draw [arrow] (v3) -- (t) node [above=0pt, sloped, pos=0.5] {$19/20$};
            \draw [arrow] (v4) -- (t) node [below=0pt, sloped, pos=0.5] {$4/4$};

        \end{tikzpicture}
    \end{figure}

    由于检验数均非负,已达最优解。

\end{example}

\begin{remark}
    在最大流的例子中,初始单纯形表中不存在单位阵,需先做一步初等行变换,也可采用两阶段单纯形法。
\end{remark}

\end{document}









