\documentclass{ctexart}
\usepackage{avanti-color}
\usepackage{avanti-font}
\usepackage{avanti-math}
\usepackage{avanti-theorem}
\usepackage{avanti-others}

\tikzset{base/.style={smooth, very thick}}
\tikzset{arrow/.style = {base, ->, -{Stealth[scale=0.8]}}}
\tikzset{aug/.style = {arrow, Solarized-blue, ultra thick}}
\tikzset{shift-above/.style = {arrow, transform canvas={yshift=0.6ex}}}
\tikzset{shift-below/.style = {arrow, transform canvas={yshift=-0.6ex}}}
\tikzset{label-above/.style = {midway, sloped, above}}
\tikzset{label-below/.style = {midway, sloped, below}}

\begin{document}
\title{最大流补充资料}
\author{张腾}
\date{}
\maketitle

Ford-Fulkerson算法的时间复杂度是$O(E ~ |f^*|)$,事实上这个界是紧的。考虑图\ref{fig: worst-case}中的流网络,其中$m$是一个很大的整数,最大流量$|f^*| = 2m$,Ford-Fulkerson算法在此流网络上需迭代$2m$次才能得到最大流。

初始化流$f = 0$,对应的残存网络就是原流网络,假设算法选择增广路径$s \rightarrow u \rightarrow v \rightarrow t$,流网络该路径上的流量都可以加$1$;接着假设选择增广路径$s \rightarrow v \rightarrow u \rightarrow t$,流网络该路径上的流量都可以加$1$;如此交替下去,每次都可以使得总流量增加$1$,总共需迭代$2m$次才能达到最优解。

\begin{figure}[h]
    \centering
    \begin{tikzpicture} [thick, scale=1, font=\normalsize]

        \pgfmathsetmacro{\x}{6}
        \pgfmathsetmacro{\y}{-3.4}
        \pgfmathsetmacro{\r}{2.5}

        \path (0,0) coordinate (cs);
        \path (cs) ++(-30:\r) coordinate (cu);
        \path (cs) ++(30:\r) coordinate (cv);
        \path (cu) ++(30:\r) coordinate (ct);
        \node (s) at (cs) {$s$};
        \node (u) at (cu) {$u$};
        \node (v) at (cv) {$v$};
        \node (t) at (ct) {$t$};

        \draw [aug] (s) -- (u) node [label-above] {$m$};
        \draw [arrow] (s) -- (v) node [label-above] {$m$};
        \draw [aug] (u) -- (v) node [label-above] {$1$};
        \draw [arrow] (u) -- (t) node [label-above] {$m$};
        \draw [aug] (v) -- (t) node [label-above] {$m$};

        \path (\x,0) coordinate (cs);
        \path (cs) ++(-30:\r) coordinate (cu);
        \path (cs) ++(30:\r) coordinate (cv);
        \path (cu) ++(30:\r) coordinate (ct);
        \node (s) at (cs) {$s$};
        \node (u) at (cu) {$u$};
        \node (v) at (cv) {$v$};
        \node (t) at (ct) {$t$};

        \draw [arrow] (s) -- (u) node [label-above] {$1 / m$};
        \draw [arrow] (s) -- (v) node [label-above] {$m$};
        \draw [arrow] (u) -- (v) node [label-above] {$1 / 1$};
        \draw [arrow] (u) -- (t) node [label-above] {$m$};
        \draw [arrow] (v) -- (t) node [label-above] {$1 / m$};

        \path (0,\y) coordinate (cs);
        \path (cs) ++(-30:\r) coordinate (cu);
        \path (cs) ++(30:\r) coordinate (cv);
        \path (cu) ++(30:\r) coordinate (ct);
        \node (s) at (cs) {$s$};
        \node (u) at (cu) {$u$};
        \node (v) at (cv) {$v$};
        \node (t) at (ct) {$t$};

        \draw [shift-above] (s) -- (u) node [label-above] {$m-1$};
        \draw [shift-below] (u) -- (s) node [label-below] {$1$};

        \draw [aug] (s) -- (v) node [label-above] {$m$};
        \draw [aug] (v) -- (u) node [label-above] {$1$};
        \draw [aug] (u) -- (t) node [label-above] {$m$};
        \draw [shift-above] (v) -- (t) node [label-above] {$m-1$};
        \draw [shift-below] (t) -- (v) node [label-below] {$1$};

        \path (\x,\y) coordinate (cs);
        \path (cs) ++(-30:\r) coordinate (cu);
        \path (cs) ++(30:\r) coordinate (cv);
        \path (cu) ++(30:\r) coordinate (ct);
        \node (s) at (cs) {$s$};
        \node (u) at (cu) {$u$};
        \node (v) at (cv) {$v$};
        \node (t) at (ct) {$t$};

        \draw [arrow] (s) -- (u) node [label-above] {$1 / m$};
        \draw [arrow] (s) -- (v) node [label-above] {$1 / m$};
        \draw [arrow] (u) -- (v) node [label-above] {$1$};
        \draw [arrow] (u) -- (t) node [label-above] {$1 / m$};
        \draw [arrow] (v) -- (t) node [label-above] {$1 / m$};

        \path (0,2*\y) coordinate (cs);
        \path (cs) ++(-30:\r) coordinate (cu);
        \path (cs) ++(30:\r) coordinate (cv);
        \path (cu) ++(30:\r) coordinate (ct);
        \node (s) at (cs) {$s$};
        \node (u) at (cu) {$u$};
        \node (v) at (cv) {$v$};
        \node (t) at (ct) {$t$};

        \draw [shift-above, aug] (s) -- (u) node [label-above] {$m-1$};
        \draw [shift-below] (u) -- (s) node [label-below] {$1$};

        \draw [shift-above] (s) -- (v) node [label-above] {$m-1$};
        \draw [shift-below] (v) -- (s) node [label-below] {$1$};

        \draw [aug] (u) -- (v) node [label-above] {$1$};

        \draw [shift-above] (u) -- (t) node [label-above] {$m-1$};
        \draw [shift-below] (t) -- (u) node [label-below] {$1$};

        \draw [shift-above, aug] (v) -- (t) node [label-above] {$m-1$};
        \draw [shift-below] (t) -- (v) node [label-below] {$1$};

        \path (\x,2*\y) coordinate (cs);
        \path (cs) ++(-30:\r) coordinate (cu);
        \path (cs) ++(30:\r) coordinate (cv);
        \path (cu) ++(30:\r) coordinate (ct);
        \node (s) at (cs) {$s$};
        \node (u) at (cu) {$u$};
        \node (v) at (cv) {$v$};
        \node (t) at (ct) {$t$};

        \draw [arrow] (s) -- (u) node [label-above] {$2 / m$};
        \draw [arrow] (s) -- (v) node [label-above] {$1 / m$};
        \draw [arrow] (u) -- (v) node [label-above] {$1 / 1$};
        \draw [arrow] (u) -- (t) node [label-above] {$1 / m$};
        \draw [arrow] (v) -- (t) node [label-above] {$2 / m$};

    \end{tikzpicture}
    \caption{每轮总流量增加$1$,总共需迭代$2m$次才能达到最大流}
    \label{fig: worst-case}
\end{figure}

\section*{无理数容量上限}

以上讨论都是针对容量上限为整数的情形,若容量上限为有理数,则先将其全部表示成既约分数,然后乘上分母的最小公倍数使其全部变成整数,之后在这个新的流网络上使用Ford-Fulkerson算法得到最大流,再除以分母的最小公倍数即可得到原流网络的最大流。

对于容量上限为无理数的情形,Ford-Fulkerson算法就无能为力了,它既不会在有限步内停止,生成的流量序列也不趋向于最大流,例如图\ref{fig: irrational-case}中的流网络,其中$m$是一个很大的整数,$\phi = (\sqrt{5} - 1) / 2$为黄金分割比,满足$1 - \phi = \phi^2$,最大流量$|f^*| = 2m+1$。

初始化流$f = 0$,对应的残存网络就是原流网络,假设算法选择增广路径$s \rightarrow b \rightarrow c \rightarrow t$,流网络该路径上的流量都可以加$1$,于是第一轮迭代的情况如下所示:

\begin{figure}[ht]
    \centering
    \begin{tikzpicture} [thick, scale=0.9, font=\normalsize]

        \pgfmathsetmacro{\r}{2.5}
        \pgfmathsetmacro{\rr}{2.4}

        \path (0,0) coordinate (ca);
        \path (\rr,0) coordinate (cb);
        \path (2*\rr,0) coordinate (cc);
        \path (3*\rr,0) coordinate (cd);

        \path (cb) ++(60:\r) coordinate (cs);
        \path (cc) ++(240:\r) coordinate (ct);

        \node (a) at (ca) {$a$};
        \node (b) at (cb) {$b$};
        \node (c) at (cc) {$c$};
        \node (d) at (cd) {$d$};
        \node (s) at (cs) {$s$};
        \node (t) at (ct) {$t$};

        \draw [arrow] (s) -- (a) node [label-above] {$m$};
        \draw [aug] (s) -- (b) node [label-above] {$m$};
        \draw [arrow] (s) -- (d) node [label-above] {$m$};

        \draw [arrow] (a) -- (t) node [label-above] {$m$};
        \draw [aug] (c) -- (t) node [label-above] {$m$};
        \draw [arrow] (d) -- (t) node [label-above] {$m$};

        \draw [arrow] (b) -- (a) node [label-above] {$1$};
        \draw [aug] (b) -- (c) node [label-above] {$1$};
        \draw [arrow] (d) -- (c) node [label-above] {$\phi$};

        \path (8.5,0) coordinate (ca);
        \path (8.5+\rr,0) coordinate (cb);
        \path (8.5+2*\rr,0) coordinate (cc);
        \path (8.5+3*\rr,0) coordinate (cd);

        \path (cb) ++(60:\r) coordinate (cs);
        \path (cc) ++(240:\r) coordinate (ct);

        \node (a) at (ca) {$a$};
        \node (b) at (cb) {$b$};
        \node (c) at (cc) {$c$};
        \node (d) at (cd) {$d$};
        \node (s) at (cs) {$s$};
        \node (t) at (ct) {$t$};

        \draw [arrow] (s) -- (a) node [label-above] {$m$};
        \draw [arrow] (s) -- (b) node [label-above] {$1 / m$};
        \draw [arrow] (s) -- (d) node [label-above] {$m$};

        \draw [arrow] (a) -- (t) node [label-above] {$m$};
        \draw [arrow] (c) -- (t) node [label-above] {$1 / m$};
        \draw [arrow] (d) -- (t) node [label-above] {$m$};

        \draw [arrow] (b) -- (a) node [label-above] {$1$};
        \draw [arrow] (b) -- (c) node [label-above] {$1/1$};
        \draw [arrow] (d) -- (c) node [label-above] {$\phi$};

    \end{tikzpicture}
    \caption{第一轮}
\end{figure}

第二轮假设算法选择的增广路径为$s \rightarrow d \rightarrow c \rightarrow b \rightarrow a \rightarrow t$,流网络该路径上的流量都可以增加$\phi$,于是

\begin{figure}[ht]
    \centering
    \begin{tikzpicture} [thick, scale=0.9, font=\normalsize]

        \pgfmathsetmacro{\r}{2.5}
        \pgfmathsetmacro{\rr}{2.4}

        \path (0,0) coordinate (ca);
        \path (\rr,0) coordinate (cb);
        \path (2*\rr,0) coordinate (cc);
        \path (3*\rr,0) coordinate (cd);

        \path (cb) ++(60:\r) coordinate (cs);
        \path (cc) ++(240:\r) coordinate (ct);

        \node (a) at (ca) {$a$};
        \node (b) at (cb) {$b$};
        \node (c) at (cc) {$c$};
        \node (d) at (cd) {$d$};
        \node (s) at (cs) {$s$};
        \node (t) at (ct) {$t$};

        \draw [arrow] (s) -- (a) node [label-above] {$m$};
        \draw [shift-above] (s) -- (b) node [label-above] {$m-1$};
        \draw [shift-below] (b) -- (s) node [midway,sloped, below] {$1$};
        \draw [aug] (s) -- (d) node [label-above] {$m$};

        \draw [aug] (a) -- (t) node [label-above] {$m$};
        \draw [shift-above] (c) -- (t) node [label-above] {$m-1$};
        \draw [shift-below] (t) -- (c) node [midway,sloped, below] {$1$};
        \draw [arrow] (d) -- (t) node [label-above] {$m$};

        \draw [aug] (b) -- (a) node [label-above] {$1$};
        \draw [aug] (c) -- (b) node [label-above] {$1$};
        \draw [aug] (d) -- (c) node [label-above] {$\phi$};

        \path (8.5+0,0) coordinate (ca);
        \path (8.5+\rr,0) coordinate (cb);
        \path (8.5+2*\rr,0) coordinate (cc);
        \path (8.5+3*\rr,0) coordinate (cd);

        \path (cb) ++(60:\r) coordinate (cs);
        \path (cc) ++(240:\r) coordinate (ct);

        \node (a) at (ca) {$a$};
        \node (b) at (cb) {$b$};
        \node (c) at (cc) {$c$};
        \node (d) at (cd) {$d$};
        \node (s) at (cs) {$s$};
        \node (t) at (ct) {$t$};

        \draw [arrow] (s) -- (a) node [label-above] {$m$};
        \draw [arrow] (s) -- (b) node [label-above] {$1 / m$};
        \draw [arrow] (s) -- (d) node [label-above] {$\phi / m$};

        \draw [arrow] (a) -- (t) node [label-above] {$\phi / m$};
        \draw [arrow] (c) -- (t) node [label-above] {$1 / m$};
        \draw [arrow] (d) -- (t) node [label-above] {$m$};

        \draw [arrow] (b) -- (a) node [label-above] {$\phi / 1$};
        \draw [arrow] (b) -- (c) node [label-above] {$1 - \phi/1$};
        \draw [arrow] (d) -- (c) node [label-above] {$\phi / \phi$};

    \end{tikzpicture}
    \caption{第二轮}
\end{figure}

结合这两个例子可以看出,容量上限为$m$的边不是瓶颈,关键是中间的三条横边,下面我们就省略源点$s$、汇点$t$以及和它们相连的边,只画中间这三条边。

\begin{figure}[!ht]
    \centering

    \pgfmathsetmacro{\x}{8}
    \pgfmathsetmacro{\y}{-1.8}
    \pgfmathsetmacro{\r}{2.5}
    \pgfmathsetmacro{\rr}{2.2}

    % iter 2

    \begin{tikzpicture} [thick, scale=1, font=\normalsize]

        \path (0,0) coordinate (ca);
        \path (\rr,0) coordinate (cb);
        \path (2*\rr,0) coordinate (cc);
        \path (3*\rr,0) coordinate (cd);

        \node (a) at (ca) {$a$};
        \node (b) at (cb) {$b$};
        \node (c) at (cc) {$c$};
        \node (d) at (cd) {$d$};

        \draw [aug] (b) -- (a) node [label-above] {$1$};
        \draw [aug] (c) -- (b) node [label-above] {$1$};
        \draw [aug] (d) -- (c) node [label-above] {$\phi$};

        \path (\x + 0,0) coordinate (ca);
        \path (\x + \rr,0) coordinate (cb);
        \path (\x + 2*\rr,0) coordinate (cc);
        \path (\x + 3*\rr,0) coordinate (cd);

        \node (a) at (ca) {$a$};
        \node (b) at (cb) {$b$};
        \node (c) at (cc) {$c$};
        \node (d) at (cd) {$d$};

        \draw [arrow] (b) -- (a) node [label-above] {$\phi / 1$};
        \draw [arrow] (b) -- (c) node [label-above] {$1 - \phi / 1$};
        \draw [arrow] (d) -- (c) node [label-above] {$\phi / \phi$};


        \path (0,    \y) coordinate (ca);
        \path (\rr,  \y) coordinate (cb);
        \path (2*\rr,\y) coordinate (cc);
        \path (3*\rr,\y) coordinate (cd);

        \node (a) at (ca) {$a$};
        \node (b) at (cb) {$b$};
        \node (c) at (cc) {$c$};
        \node (d) at (cd) {$d$};

        \draw [shift-above] (b) -- (a) node [label-above] {$\phi^2$};
        \draw [shift-below] (a) -- (b) node [label-below] {$1 - \phi^2$};
        \draw [shift-above, aug] (b) -- (c) node [label-above] {$\phi$};
        \draw [shift-below] (c) -- (b) node [label-below] {$1 - \phi$};
        \draw [aug] (c) -- (d) node [label-above] {$\phi$};

        \path (\x + 0,    \y) coordinate (ca);
        \path (\x + \rr,  \y) coordinate (cb);
        \path (\x + 2*\rr,\y) coordinate (cc);
        \path (\x + 3*\rr,\y) coordinate (cd);

        \node (a) at (ca) {$a$};
        \node (b) at (cb) {$b$};
        \node (c) at (cc) {$c$};
        \node (d) at (cd) {$d$};

        \draw [arrow] (b) -- (a) node [label-above] {$\phi / 1$};
        \draw [arrow] (b) -- (c) node [label-above] {$1 / 1$};
        \draw [arrow] (d) -- (c) node [label-above] {$\phi$};


        \path (0,    2*\y) coordinate (ca);
        \path (\rr,  2*\y) coordinate (cb);
        \path (2*\rr,2*\y) coordinate (cc);
        \path (3*\rr,2*\y) coordinate (cd);

        \node (a) at (ca) {$a$};
        \node (b) at (cb) {$b$};
        \node (c) at (cc) {$c$};
        \node (d) at (cd) {$d$};

        \draw [shift-above, aug] (b) -- (a) node [label-above] {$\phi^2$};
        \draw [shift-below] (a) -- (b) node [label-below] {$1 - \phi^2$};
        \draw [aug] (c) -- (b) node [label-above] {$1$};
        \draw [aug] (d) -- (c) node [label-above] {$\phi$};

        \path (\x + 0,     2 * \y) coordinate (ca);
        \path (\x + \rr,   2 * \y) coordinate (cb);
        \path (\x + 2*\rr, 2 * \y) coordinate (cc);
        \path (\x + 3*\rr, 2 * \y) coordinate (cd);

        \node (a) at (ca) {$a$};
        \node (b) at (cb) {$b$};
        \node (c) at (cc) {$c$};
        \node (d) at (cd) {$d$};

        \draw [arrow] (b) -- (a) node [label-above] {$1 / 1$};
        \draw [arrow] (b) -- (c) node [label-above] {$1 - \phi^2 / 1$};
        \draw [arrow] (d) -- (c) node [label-above] {$\phi^2 / \phi$};

        \path (0,     3 * \y) coordinate (ca);
        \path (\rr,   3 * \y) coordinate (cb);
        \path (2*\rr, 3 * \y) coordinate (cc);
        \path (3*\rr, 3 * \y) coordinate (cd);

        \node (a) at (ca) {$a$};
        \node (b) at (cb) {$b$};
        \node (c) at (cc) {$c$};
        \node (d) at (cd) {$d$};

        \draw [aug] (a) -- (b) node [label-above] {$1$};
        \draw [shift-below] (c) -- (b) node [label-below] {$1 - \phi^2$};
        \draw [shift-above, aug] (b) -- (c) node [label-above] {$\phi^2$};
        \draw [shift-above] (d) -- (c) node [label-above] {$\phi^3$};
        \draw [shift-below] (c) -- (d) node [label-below] {$\phi - \phi^3$};

        \path (\x + 0,     3 * \y) coordinate (ca);
        \path (\x + \rr,   3 * \y) coordinate (cb);
        \path (\x + 2*\rr, 3 * \y) coordinate (cc);
        \path (\x + 3*\rr, 3 * \y) coordinate (cd);

        \node (a) at (ca) {$a$};
        \node (b) at (cb) {$b$};
        \node (c) at (cc) {$c$};
        \node (d) at (cd) {$d$};

        \draw [arrow] (b) -- (a) node [label-above] {$1 - \phi^2 / 1$};
        \draw [arrow] (b) -- (c) node [label-above] {$1 / 1$};
        \draw [arrow] (d) -- (c) node [label-above] {$\phi^2 / \phi$};


        \path (0,     4 * \y) coordinate (ca);
        \path (\rr,   4 * \y) coordinate (cb);
        \path (2*\rr, 4 * \y) coordinate (cc);
        \path (3*\rr, 4 * \y) coordinate (cd);

        \node (a) at (ca) {$a$};
        \node (b) at (cb) {$b$};
        \node (c) at (cc) {$c$};
        \node (d) at (cd) {$d$};

        \draw [shift-above, aug] (b) -- (a) node [label-above] {$\phi^2$};
        \draw [shift-below] (a) -- (b) node [label-below] {$1 - \phi^2$};
        \draw [aug] (c) -- (b) node [label-above] {$1$};
        \draw [shift-above, aug] (d) -- (c) node [label-above] {$\phi^3$};
        \draw [shift-below] (c) -- (d) node [label-below] {$\phi - \phi^3$};

        \path (\x + 0,     4 * \y) coordinate (ca);
        \path (\x + \rr,   4 * \y) coordinate (cb);
        \path (\x + 2*\rr, 4 * \y) coordinate (cc);
        \path (\x + 3*\rr, 4 * \y) coordinate (cd);

        \node (a) at (ca) {$a$};
        \node (b) at (cb) {$b$};
        \node (c) at (cc) {$c$};
        \node (d) at (cd) {$d$};

        \draw [arrow] (b) -- (a) node [label-above] {$\phi + \phi^3 / 1$};
        \draw [arrow] (b) -- (c) node [label-above] {$1 - \phi^3 / 1$};
        \draw [arrow] (d) -- (c) node [label-above] {$\phi / \phi$};


        \path (0,     5 * \y) coordinate (ca);
        \path (\rr,   5 * \y) coordinate (cb);
        \path (2*\rr, 5 * \y) coordinate (cc);
        \path (3*\rr, 5 * \y) coordinate (cd);

        \node (a) at (ca) {$a$};
        \node (b) at (cb) {$b$};
        \node (c) at (cc) {$c$};
        \node (d) at (cd) {$d$};

        \draw [shift-above] (b) -- (a) node [label-above] {$\phi^4$};
        \draw [shift-below] (a) -- (b) node [label-below] {$1 - \phi^4$};
        \draw [shift-above, aug] (b) -- (c) node [label-above] {$\phi^3$};
        \draw [shift-below] (c) -- (b) node [label-below] {$1 - \phi^3$};
        \draw [aug] (c) -- (d) node [label-above] {$\phi$};

        \path (\x + 0,     5 * \y) coordinate (ca);
        \path (\x + \rr,   5 * \y) coordinate (cb);
        \path (\x + 2*\rr, 5 * \y) coordinate (cc);
        \path (\x + 3*\rr, 5 * \y) coordinate (cd);

        \node (a) at (ca) {$a$};
        \node (b) at (cb) {$b$};
        \node (c) at (cc) {$c$};
        \node (d) at (cd) {$d$};

        \draw [arrow] (b) -- (a) node [label-above] {$\phi + \phi^3 / 1$};
        \draw [arrow] (b) -- (c) node [label-above] {$1 / 1$};
        \draw [arrow] (d) -- (c) node [label-above] {$\phi - \phi^3 / \phi$};


        \path (0,     6 * \y) coordinate (ca);
        \path (\rr,   6 * \y) coordinate (cb);
        \path (2*\rr, 6 * \y) coordinate (cc);
        \path (3*\rr, 6 * \y) coordinate (cd);

        \node (a) at (ca) {$a$};
        \node (b) at (cb) {$b$};
        \node (c) at (cc) {$c$};
        \node (d) at (cd) {$d$};

        \draw [shift-above, aug] (b) -- (a) node [label-above] {$\phi^4$};
        \draw [shift-below] (a) -- (b) node [label-below] {$1 - \phi^4$};
        \draw [aug] (c) -- (b) node [label-above] {$1$};
        \draw [shift-above, aug] (d) -- (c) node [label-above] {$\phi^3$};
        \draw [shift-below] (c) -- (d) node [label-below] {$\phi - \phi^3$};

        \path (\x + 0,     6 * \y) coordinate (ca);
        \path (\x + \rr,   6 * \y) coordinate (cb);
        \path (\x + 2*\rr, 6 * \y) coordinate (cc);
        \path (\x + 3*\rr, 6 * \y) coordinate (cd);

        \node (a) at (ca) {$a$};
        \node (b) at (cb) {$b$};
        \node (c) at (cc) {$c$};
        \node (d) at (cd) {$d$};

        \draw [arrow] (b) -- (a) node [label-above] {$1 / 1$};
        \draw [arrow] (b) -- (c) node [label-above] {$1 - \phi^4 / 1$};
        \draw [arrow] (d) -- (c) node [label-above] {$\phi^2 + \phi^4 / \phi$};


        \path (0,     7 * \y) coordinate (ca);
        \path (\rr,   7 * \y) coordinate (cb);
        \path (2*\rr, 7 * \y) coordinate (cc);
        \path (3*\rr, 7 * \y) coordinate (cd);

        \node (a) at (ca) {$a$};
        \node (b) at (cb) {$b$};
        \node (c) at (cc) {$c$};
        \node (d) at (cd) {$d$};

        \draw [aug] (a) -- (b) node [label-above] {$1$};
        \draw [shift-above, aug] (b) -- (c) node [label-above] {$\phi^4$};
        \draw [shift-below] (c) -- (b) node [label-below] {$1 - \phi^4$};
        \draw [shift-above] (d) -- (c) node [label-above] {$\phi^5$};
        \draw [shift-below] (c) -- (d) node [label-below] {$\phi - \phi^5$};

        \path (\x + 0,     7 * \y) coordinate (ca);
        \path (\x + \rr,   7 * \y) coordinate (cb);
        \path (\x + 2*\rr, 7 * \y) coordinate (cc);
        \path (\x + 3*\rr, 7 * \y) coordinate (cd);

        \node (a) at (ca) {$a$};
        \node (b) at (cb) {$b$};
        \node (c) at (cc) {$c$};
        \node (d) at (cd) {$d$};

        \draw [arrow] (b) -- (a) node [label-above] {$1 - \phi^4 / 1$};
        \draw [arrow] (b) -- (c) node [label-above] {$1 / 1$};
        \draw [arrow] (d) -- (c) node [label-above] {$\phi^2 + \phi^4 / \phi$};

    \end{tikzpicture}
    \caption{第二轮至第九轮,九轮后流值$|f| = 1 + 2 \phi + 2 \phi^2 + 2 \phi^3 + 2 \phi^4$}

\end{figure}

不难看出,每$4$轮一个周期,经过$4 n + 1$次迭代后,总流量
\begin{align*}
    |f| = 1 + 2 \sum_{i=1}^{2n} \phi^i = 1 + 2 \frac{\phi(1 - \phi^{2n})}{1 - \phi} \xrightarrow{\quad n \rightarrow \infty \quad} 1 + 2 \frac{\phi}{1 - \phi} = 1 + \frac{2}{\phi} = 2 + \sqrt{5} \ll 2m+1
\end{align*}

\end{document}

