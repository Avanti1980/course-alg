\documentclass{ctexart}
\usepackage{avanti}

\begin{document}
\title{\textbf{主定理证明}}
\author{张腾}
\date{\today}
\maketitle

\begin{theorem} [主定理]
    令$a \ge 1$和$b > 1$是常数,$f(n)$是一个函数,$T(n)$是定义在\textbf{非负整数}上的递归式:
    \begin{align*}
        T(n) = a \cdot T \left( \frac{n}{b} \right) + f(n)
    \end{align*}
    其中我们将$n/b$解释为$\lfloor n/b \rfloor$或$\lceil n/b \rceil$,那么$T(n)$有如下渐进界:
    \begin{enumerate}
        \item 若对某个常数$\varepsilon > 0$有$f(n) = O(n^{\log_b a - \varepsilon})$,则$T(n) = \Theta (n^{\log_b a})$。
        \item 若$f(n) = \Theta (n^{\log_b a})$,则$T(n) = \Theta (n^{\log_b a} \lg n)$。
        \item 若对某个常数$\varepsilon > 0$有$f(n) = \Omega(n^{\log_b a + \varepsilon})$,且对某个常数$c<1$和所有足够大的$n$有$a f(n/b) \le c f(n)$,则$T(n) = \Theta (f(n))$。
    \end{enumerate}
\end{theorem}

\begin{proof}
    若$n$是$b$的整数幂次,不妨设$n = b^k$,此时$n/b$最后正好会递归成$1$,即
    \begin{align*}
        T \left( \frac{n}{b^0} \right)     & = a \cdot T \left( \frac{n}{b^1} \right) + f \left( \frac{n}{b^0} \right)     \\
        T \left( \frac{n}{b^1} \right)     & = a \cdot T \left( \frac{n}{b^2} \right) + f \left( \frac{n}{b^1} \right)     \\
                                           & \vdots                                                                        \\
        T \left( \frac{n}{b^{k-1}} \right) & = a \cdot T \left( \frac{n}{b^k} \right) + f \left( \frac{n}{b^{k-1}} \right)
    \end{align*}
    注意上面共有$k$个递推式,又$a^k = (b^{\log_b a})^k = (b^k)^{\log_b a} = n^{\log_b a}$、$n = b^k$,于是
    \begin{align*}
        T(n) = a^k \cdot T \left( \frac{n}{b^k} \right) + f \left( \frac{n}{b^0} \right) + a f \left( \frac{n}{b^1} \right) + \cdots + a^{k-1} f \left( \frac{n}{b^{k-1}} \right) = n^{\log_b a} \cdot T(1) + \sum_{i=0}^{k-1} a^i f \left( \frac{n}{b^i} \right)
    \end{align*}
    显然前一项$n^{\log_b a} \cdot T(1) = \Theta(n^{\log_b a})$,故关键就是后面求和这一项,下面分情况讨论:
    \begin{enumerate}
        \item 由$f(n) = O(n^{\log_b a - \varepsilon})$可知
              \begin{align*}
                  a^i f \left( \frac{n}{b^i} \right) = a^i O \left( \left( \frac{n}{b^i} \right)^{\log_b a - \varepsilon} \right) = a^i O \left( \frac{n^{\log_b a - \varepsilon}}{a^i b^{-\varepsilon i}} \right) = O(n^{\log_b a - \varepsilon}) b^{\varepsilon i}
              \end{align*}
              故
              \begin{align*}
                  g(n) = \sum_{i=0}^{k-1} f \left( \frac{n}{b^i} \right) = O(n^{\log_b a - \varepsilon}) \sum_{i=0}^{k-1} b^{\varepsilon i} = O(n^{\log_b a - \varepsilon}) \frac{1 - b^{k \varepsilon}}{1 - b^\varepsilon} = O(n^{\log_b a - \varepsilon}) \frac{n^\varepsilon - 1}{b^\varepsilon - 1} = O(n^{\log_b a})
              \end{align*}
              从而$T(n) = \Theta(n^{\log_b a}) + O(n^{\log_b a}) = \Theta(n^{\log_b a})$,即此时$T(n)$的复杂度由前一项决定。

        \item 由$f(n) = \Theta (n^{\log_b a})$可知
              \begin{align*}
                  a^i f \left( \frac{n}{b^i} \right) = a^i \Theta \left( \left( \frac{n}{b^i} \right)^{\log_b a} \right) = a^i \Theta \left( \frac{n^{\log_b a}}{a^i} \right) = \Theta(n^{\log_b a})
              \end{align*}
              于是
              \begin{align*}
                  g(n) = \sum_{i=0}^{k-1} a^i f \left( \frac{n}{b^i} \right) = \sum_{i=0}^{k-1} \Theta(n^{\log_b a}) = \Theta(n^{\log_b a}) \log_b n = \Theta(n^{\log_b a} \lg n)
              \end{align*}
              从而$T(n) = \Theta(n^{\log_b a}) + \Theta(n^{\log_b a} \lg n) = \Theta(n^{\log_b a} \lg n)$,即此时$T(n)$的复杂度由求和这一项决定。

        \item 由条件知存在$j$使得当$n' \ge b^{k-j+1} = n / b^{j-1}$时递推式$a f(n'/b) \le c f(n')$恒成立,于是
              \begin{align*}
                  a^j f \left( \frac{n}{b^j} \right) \le a^{j-1} c f \left( \frac{n}{b^{j-1}} \right) \le \cdots \le a^2 c^{j-2} f \left( \frac{n}{b^2} \right) \le a c^{j-1} f \left( \frac{n}{b} \right) \le c^j f(n)
              \end{align*}
              从而对$\forall i \le j$有$a^i f(n/b^i) \le c^{i} f(n)$,代入可得
              \begin{align*}
                  g(n) & = \sum_{i=0}^{k-1} a^i f \left( \frac{n}{b^i} \right) = \sum_{i=0}^j a^i f \left( \frac{n}{b^i} \right) + \overbrace{a^{j+1} f \left(\frac{n}{b^{j+1}} \right) + \cdots + a^{k-1} f \left(\frac{n}{b^{k-1}} \right)}^{\text{不满足递推式的部分,复杂度为}O(1)} \\
                       & \le \sum_{i=0}^j c^i f(n) + O(1) < f(n) \sum_{i=0}^\infty c^i + O(1) \overset{c<1}{=} \frac{f(n)}{1-c} + O(1) = O(f(n))
              \end{align*}
              又$f(n)$是$g(n)$中的一项且$g(n)$所有求和项非负,故$g(n) \ge f(n)$,从而
              \begin{align*}
                  g(n) = \Theta (f(n)) = \Omega(n^{\log_b a + \varepsilon}) > \Theta(n^{\log_b a})
              \end{align*}
              于是$T(n) = \Theta(n^{\log_b a}) + g(n) = \Theta (f(n))$,即此时$T(n)$的复杂度由求和这一项决定。
    \end{enumerate}

    \noindent \rule[0.25\baselineskip]{\textwidth}{0.5pt}

    若$n$不是$b$的整数幂次,则递推中某一轮$n/b$不一定为整数,又$T(n)$是定义在非负整数上的,因此还需多做一步取整,此时递推式可写成
    \begin{align*}
        T(n_0) & = a \cdot T(n_1) + f(n_0), \quad n_1 = [n_0/b] \\
        T(n_1) & = a \cdot T(n_2) + f(n_1), \quad n_2 = [n_1/b] \\
        T(n_2) & = a \cdot T(n_3) + f(n_2), \quad n_3 = [n_2/b] \\
               & \vdots
    \end{align*}
    不妨先考虑向上取整,即
    \begin{align*}
        n_i = \begin{cases}
                  n & i = 0 \\ \lceil n_{i-1} / b \rceil & i > 0
              \end{cases}
    \end{align*}
    根据$x \le \lceil x \rceil \le x + 1$易知
    \begin{align*}
        \frac{n}{b}   \le n_1 & \le \frac{n}{b} + 1                                 \\
        \frac{n}{b^2} \le n_2 & \le \frac{n}{b^2} + \frac{1}{b} + 1                 \\
        \frac{n}{b^3} \le n_3 & \le \frac{n}{b^3} + \frac{1}{b^2} + \frac{1}{b} + 1 \\
        \vdots                &
    \end{align*}
    于是
    \begin{align} \label{eq: ni-upper-bound}
        \frac{n}{b^i} \le n_i \le \frac{n}{b^i} + \frac{1}{b^{i-1}} + \cdots + \frac{1}{b} + 1 < \frac{n}{b^i} + \frac{1}{1 - 1/b} = \frac{n}{b^i} + \frac{b}{b-1}
    \end{align}
    令$k = \lfloor \log_b n \rfloor \ge \log_b n - 1$,即$n \le b^{k+1}$,代入式(\ref{eq: ni-upper-bound})可得
    \begin{align*}
        n_k < \frac{n}{b^k} + \frac{b}{b-1} \le \frac{b^{k+1}}{b^k} + \frac{b}{b-1} = b + \frac{b}{b-1} = O(1)
    \end{align*}
    故递推到第$k$轮,子问题$T(n_k)$已经是$O(1)$复杂度了,同前面一样有
    \begin{align*}
        T(n) = a^k T(n_k) + f(n_0) + a f(n_1) + \cdots + a^{k-1} f(n_{k-1}) = \Theta(n^{\log_b a}) + \sum_{i=0}^{k-1} a^i f(n_i)
    \end{align*}
    关键依然是后面求和这一项,继续分情况讨论:
    \begin{enumerate}
        \item 由$f(n) = O(n^{\log_b a - \varepsilon})$可知存在正常数$c_i$和$n_i$使得对$\forall n \ge n_i$有
              \begin{align*}
                  a^i f(n_i) & \le a^i c_i n_i^{\log_b a - \varepsilon} < a^i c_i \left( \frac{n}{b^i} + \frac{b}{b-1} \right)^{\log_b a - \varepsilon} = a^i c_i \left( \frac{n}{b^i} \right)^{\log_b a - \varepsilon} \left( 1 + \frac{b^i}{n} \frac{b}{b-1} \right)^{\log_b a - \varepsilon} \\
                             & \overset{b^i \le n}{\le} a^i c_i n^{\log_b a - \varepsilon} \frac{b^{\varepsilon i}}{a^i} \left( 1 + \frac{b}{b-1} \right)^{\log_b a - \varepsilon} = c_i n^{\log_b a - \varepsilon} b^{\varepsilon i} \left( \frac{2b-1}{b-1} \right)^{\log_b a - \varepsilon}
              \end{align*}
              令$\cbar = \max_i \{c_i\}$、$\nbar = \max_i \{n_i\}$,于是对$\forall n \ge \nbar$有
              \begin{align*}
                  g(n) & = \sum_{i=0}^{k-1} a^i f(n_i) < \sum_{i=0}^{k-1} c_i n^{\log_b a - \varepsilon} b^{\varepsilon i} \left( \frac{2b-1}{b-1} \right)^{\log_b a - \varepsilon} \le \cbar n^{\log_b a - \varepsilon} \left( \frac{2b-1}{b-1} \right)^{\log_b a - \varepsilon} \sum_{i=0}^{k-1} b^{\varepsilon i} \\
                       & \overset{b^k \le n}{\le} \cbar n^{\log_b a - \varepsilon} \left( \frac{2b-1}{b-1} \right)^{\log_b a - \varepsilon} \frac{n^\varepsilon - 1}{b^\varepsilon - 1} < \frac{\cbar}{b^\varepsilon - 1} \left( \frac{2b-1}{b-1} \right)^{\log_b a - \varepsilon} n^{\log_b a} = O(n^{\log_b a})
              \end{align*}
              从而$T(n) = \Theta(n^{\log_b a}) + O(n^{\log_b a}) = \Theta(n^{\log_b a})$,即此时$T(n)$的复杂度由前一项决定。

        \item 由$f(n) = \Theta (n^{\log_b a})$可知存在正常数$c_i$、$d_i$和$n_i$使得对$\forall n \ge n_i$有
              \begin{align*}
                  a^i f(n_i) \le a^i c_i n_i^{\log_b a} < a^i c_i \left( \frac{n}{b^i} \right)^{\log_b a} \left( 1 + \frac{b^i}{n} \frac{b}{b-1} \right)^{\log_b a} \le c_i \left( 1 + \frac{b}{b-1} \right)^{\log_b a} n^{\log_b a} = O(n^{\log_b a})
              \end{align*}
              以及
              \begin{align*}
                  a^i f(n_i) \ge a^i d_i n_i^{\log_b a} \ge a^i d_i \left( \frac{n}{b^i} \right)^{\log_b a} = d_i n^{\log_b a} = \Omega (n^{\log_b a})
              \end{align*}
              故$a^i f(n_i) = \Theta(n^{\log_b a})$,于是
              \begin{align*}
                  g(n) = \sum_{i=0}^{k-1} a^i f(n_i) = \Theta(n^{\log_b a}) k = \Theta(n^{\log_b a} \lg n)
              \end{align*}
              从而$T(n) = \Theta(n^{\log_b a}) + \Theta(n^{\log_b a} \lg n) = \Theta(n^{\log_b a} \lg n)$,即此时$T(n)$的复杂度由求和这一项决定。

        \item 由条件知存在$j$使得当$n \ge n_{j-1}$时递推式$a f(\lceil n/b \rceil) \le c f(n)$恒成立,于是
              \begin{align*}
                  a^j f(n_j) \le a^{j-1} c f(n_{j-1}) \le \cdots \le a^2 c^{j-2} f(n_2) \le a c^{j-1} f(n_1) \le c^j f(n_0)
              \end{align*}
              从而对$\forall i \le j$有$a^i f(n_i) \le c^{i} f(n)$,代入可得
              \begin{align*}
                  g(n) & = \sum_{i=0}^{k-1} a^i f(n_i) = \sum_{i=0}^j a^i f(n_i) + \overbrace{a^{j+1} f(n_{j+1}) + \cdots + a^{k-1} f(n_{k-1})}^{\text{不满足递推式的部分,复杂度为}O(1)} \\
                       & \le \sum_{i=0}^j c^i f(n) + O(1) < f(n) \sum_{i=0}^\infty c^i + O(1) \overset{c<1}{=} \frac{f(n)}{1-c} + O(1) = O(f(n))
              \end{align*}
              又$f(n)$是$g(n)$中的一项且$g(n)$所有求和项非负,故$g(n) \ge f(n)$,从而
              \begin{align*}
                  g(n) = \Theta (f(n)) = \Omega(n^{\log_b a + \varepsilon}) > \Theta(n^{\log_b a})
              \end{align*}
              于是$T(n) = \Theta(n^{\log_b a}) + g(n) = \Theta (f(n))$,即此时$T(n)$的复杂度由求和这一项决定。
    \end{enumerate}

    \noindent \rule[0.25\baselineskip]{\textwidth}{0.5pt}

    最后考虑向下取整,即
    \begin{align*}
        n_i = \begin{cases}
                  n & i = 0 \\ \lfloor n_{i-1}/b \rfloor & i > 0
              \end{cases}
    \end{align*}
    根据$x-1 \le \lfloor x \rfloor \le x$易知
    \begin{align*}
        \frac{n}{b} - 1   \le n_1                               & \le \frac{n}{b}   \\
        \frac{n}{b^2} - \frac{1}{b} - 1    \le n_2              & \le \frac{n}{b^2} \\
        \frac{n}{b^3} - \frac{1}{b^2} - \frac{1}{b} - 1 \le n_3 & \le \frac{n}{b^3} \\
        \vdots                                                  &
    \end{align*}
    于是
    \begin{align} \label{eq: ni-lower-bound}
        \frac{n}{b^i} - \frac{b}{b-1} = \frac{n}{b^i} - \frac{1}{1 - 1/b} < \frac{n}{b^i} - \frac{1}{b^{i-1}} - \cdots - \frac{1}{b} - 1 \le n_i \le \frac{n}{b^i}
    \end{align}
    令$k = \lfloor \log_b n(b-1) \rfloor - 2 \ge \log_b n(b-1) - 3$,即$n(b-1) \le b^{k+3}$,代入式(\ref{eq: ni-lower-bound})可得
    \begin{align*}
        n_k \le \frac{n}{b^k} \le \frac{b^{k+3}}{b^k} = \frac{b^3}{b-1} = O(1)
    \end{align*}
    故递推到第$k$轮,子问题$T(n_k)$已经是$O(1)$复杂度了,同前面一样有
    \begin{align*}
        T(n) = a^k T(n_k) + f(n_0) + a f(n_1) + \cdots + a^{k-1} f(n_{k-1}) = \Theta(n^{\log_b a}) + \sum_{i=0}^{k-1} a^i f(n_i)
    \end{align*}
    关键依然是后面求和这一项,继续分情况讨论:
    \begin{enumerate}
        \item 由$f(n) = O(n^{\log_b a - \varepsilon})$可知存在正常数$c_i$和$n_i$使得对$\forall n \ge n_i$有
              \begin{align*}
                  a^i f(n_i) \le a^i c_i n_i^{\log_b a - \varepsilon} < a^i c_i \left( \frac{n}{b^i} \right)^{\log_b a - \varepsilon} = c_i n^{\log_b a - \varepsilon} b^{\varepsilon i}
              \end{align*}
              令$\cbar = \max_i \{c_i\}$、$\nbar = \max_i \{n_i\}$,于是对$\forall n \ge \nbar$有
              \begin{align*}
                  g(n) & = \sum_{i=0}^{k-1} a^i f(n_i) < \sum_{i=0}^{k-1} c_i n^{\log_b a - \varepsilon} b^{\varepsilon i} \le \cbar n^{\log_b a - \varepsilon} \sum_{i=0}^{k-1} b^{\varepsilon i} \le \cbar n^{\log_b a - \varepsilon} \frac{n^\varepsilon - 1}{b^\varepsilon - 1} = O(n^{\log_b a})
              \end{align*}
              从而$T(n) = \Theta(n^{\log_b a}) + O(n^{\log_b a}) = \Theta(n^{\log_b a})$,即此时$T(n)$的复杂度由前一项决定。

        \item 由$f(n) = \Theta (n^{\log_b a})$可知存在正常数$c_i$、$d_i$和$n_i$使得对$\forall n \ge n_i$有
              \begin{align*}
                  a^i f(n_i) \le a^i c_i n_i^{\log_b a} \le a^i c_i \left( \frac{n}{b^i} \right)^{\log_b a} \le c_i n^{\log_b a} = O(n^{\log_b a})
              \end{align*}
              以及
              \begin{align*}
                  a^i f(n_i) & \ge a^i d_i n_i^{\log_b a} \ge a^i d_i \left( \frac{n}{b^i} - \frac{b}{b-1} \right)^{\log_b a} = a^i d_i \left( \frac{n}{b^i} \right)^{\log_b a} \left( 1 - \frac{b^i}{n} \frac{b}{b-1} \right)^{\log_b a} \\
                             & = d_i n^{\log_b a} \left( 1 - \frac{b^i}{n} \frac{b}{b-1} \right)^{\log_b a}
              \end{align*}
              注意$k = \lfloor \log_b n(b-1) \rfloor - 2 \le \log_b n(b-1) - 2$,即$n(b-1) \ge b^{k+2}$,于是
              \begin{align*}
                  1 > 1 - \frac{b^i}{n} \frac{b}{b-1} \ge 1 - \frac{b^{i+1}}{b^{k+2}} > 0 \Longrightarrow 1 - \frac{b^i}{n} \frac{b}{b-1} \in (0,1) \Longrightarrow a^i f(n_i) = \Omega (n^{\log_b a})
              \end{align*}
              故$a^i f(n_i) = \Theta(n^{\log_b a})$,于是
              \begin{align*}
                  g(n) = \sum_{i=0}^{k-1} a^i f(n_i) = \Theta(n^{\log_b a}) k = \Theta(n^{\log_b a} \lg n)
              \end{align*}
              从而$T(n) = \Theta(n^{\log_b a}) + \Theta(n^{\log_b a} \lg n) = \Theta(n^{\log_b a} \lg n)$,即此时$T(n)$的复杂度由求和这一项决定。

        \item 由条件知存在$j$使得当$n \ge n_{j-1}$时递推式$a f(\lfloor n/b \rfloor) \le c f(n)$恒成立,于是
              \begin{align*}
                  a^j f(n_j) \le a^{j-1} c f(n_{j-1}) \le \cdots \le a^2 c^{j-2} f(n_2) \le a c^{j-1} f(n_1) \le c^j f(n_0)
              \end{align*}
              从而对$\forall i \le j$有$a^i f(n_i) \le c^{i} f(n)$,代入可得
              \begin{align*}
                  g(n) & = \sum_{i=0}^{k-1} a^i f(n_i) = \sum_{i=0}^j a^i f(n_i) + \overbrace{a^{j+1} f(n_{j+1}) + \cdots + a^{k-1} f(n_{k-1})}^{\text{不满足递推式的部分,复杂度为}O(1)} \\
                       & \le \sum_{i=0}^j c^i f(n) + O(1) < f(n) \sum_{i=0}^\infty c^i + O(1) \overset{c<1}{=} \frac{f(n)}{1-c} + O(1) = O(f(n))
              \end{align*}
              又$f(n)$是$g(n)$中的一项且$g(n)$所有求和项非负,故$g(n) \ge f(n)$,从而
              \begin{align*}
                  g(n) = \Theta (f(n)) = \Omega(n^{\log_b a + \varepsilon}) > \Theta(n^{\log_b a})
              \end{align*}
              于是$T(n) = \Theta(n^{\log_b a}) + g(n) = \Theta (f(n))$,即此时$T(n)$的复杂度由求和这一项决定。
    \end{enumerate}


\end{proof}


\end{document}
