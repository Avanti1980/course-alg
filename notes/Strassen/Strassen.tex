\documentclass{ctexart}
\usepackage{avanti-color}
\usepackage{avanti-font}
\usepackage{avanti-math}
\usepackage{avanti-theorem}
\usepackage{avanti-others}

\begin{document}
\title{\textbf{矩阵乘法加速}}
\author{张腾}
\date{\today}
\maketitle

设矩阵$\Av = (a_{ij})$和$\Bv = (b_{ij})$是$n$阶方阵,乘积$\Cv = (c_{ij})$亦是$n$阶方阵,其中
\begin{align*}
    c_{ij} = \sum_{k \in [n]} a_{ik} b_{kj}
\end{align*}
因此按标准的矩阵乘法,计算$\Cv$的时间开销为$\Omega(n^3)$,事实上这个时间复杂度是可以改进的。

\section{分治递归}

设计算$\Cv$的时间开销为$T(n)$,将矩阵分成$2 \times 2 = 4$块,由分块矩阵乘法有
\begin{align*}
    \begin{bmatrix}
        \Cv_{11} & \Cv_{12} \\ \Cv_{21} & \Cv_{22}
    \end{bmatrix} =
    \begin{bmatrix}
        \Av_{11} & \Av_{12} \\ \Av_{21} & \Av_{22}
    \end{bmatrix}
    \begin{bmatrix}
        \Bv_{11} & \Bv_{12} \\ \Bv_{21} & \Bv_{22}
    \end{bmatrix} =
    \begin{bmatrix}
        \Av_{11} \Bv_{11} + \Av_{12} \Bv_{21} & \Av_{11} \Bv_{12} + \Av_{12} \Bv_{22} \\
        \Av_{21} \Bv_{11} + \Av_{22} \Bv_{21} & \Av_{21} \Bv_{12} + \Av_{22} \Bv_{22}
    \end{bmatrix}
\end{align*}
其中包含$8$个$n/2$阶方阵相乘、$4$个$n/2$阶方阵相加,注意每个$n/2$阶方阵有$n^2/4$个元素,因此共需进行$n^2$次加法,综上有递推关系
\begin{align*}
    T(n) = 8 \cdot T(n/2) + c_1 n^2
\end{align*}
其中$c_1$为单次加法的时间开销。设$n = 2^k$,则
\begin{align*}
    T(2^k)               & = 8^1 \cdot T(2^{k-1}) + 8^0 \cdot c_1 4^k     \\
    8^1 \cdot T(2^{k-1}) & = 8^2 \cdot T(2^{k-2}) + 8^1 \cdot c_1 4^{k-1} \\
    8^2 \cdot T(2^{k-2}) & = 8^3 \cdot T(2^{k-3}) + 8^2 \cdot c_1 4^{k-2} \\
                         & \vdots                                         \\
    8^{k-1} \cdot T(2^1) & = 8^k \cdot T(2^0) + 8^{k-1} \cdot c_1 4^1
\end{align*}
注意$8^k = n^3$,$T(1) = c_2$是单次乘法的时间开销,累加可得
\begin{align*}
    T(n) & = c_2 n^3 + c_1 4^k + 2^1 \cdot c_1 4^k + 2^2 \cdot c_1 4^k + \cdots + 2^{k-1} \cdot c_1 4^k = c_2 n^3 + c_1 4^k \frac{1-2^k}{1-2} \\
         & = c_2 n^3 + c_1 n^2 (n-1) = (c_2 + c_1) n^3 - c_1 n^2
\end{align*}
即\blue{直接分治递归并不能改进时间复杂度}。

\section{基本想法}

要想改进时间复杂度,必须得减少子问题的个数,即乘法的次数。将乘积拉直,易知
\begin{align*}
    \begin{bmatrix}
        \Av_{11} \Bv_{11} + \Av_{12} \Bv_{21} \\
        \Av_{11} \Bv_{12} + \Av_{12} \Bv_{22} \\
        \Av_{21} \Bv_{11} + \Av_{22} \Bv_{21} \\
        \Av_{21} \Bv_{12} + \Av_{22} \Bv_{22}
    \end{bmatrix} =
    \begin{bmatrix}
        \Av_{11} & \zerov   & \Av_{12} & \zerov   \\
        \zerov   & \Av_{11} & \zerov   & \Av_{12} \\
        \Av_{21} & \zerov   & \Av_{22} & \zerov   \\
        \zerov   & \Av_{21} & \zerov   & \Av_{22}
    \end{bmatrix}
    \begin{bmatrix}
        \Bv_{11} \\ \Bv_{12} \\ \Bv_{21} \\ \Bv_{22}
    \end{bmatrix} = \widetilde{\Av}
    \begin{bmatrix}
        \Bv_{11} \\ \Bv_{12} \\ \Bv_{21} \\ \Bv_{22}
    \end{bmatrix}
\end{align*}
\blue{现假设$\widetilde{\Av}$可以分解成$m$个``块秩$1$矩阵''的和:
    \begin{align} \label{eq: decomposition}
        \widetilde{\Av} =
        \begin{bmatrix}
            \Av_{11} & \zerov   & \Av_{12} & \zerov   \\
            \zerov   & \Av_{11} & \zerov   & \Av_{12} \\
            \Av_{21} & \zerov   & \Av_{22} & \zerov   \\
            \zerov   & \Av_{21} & \zerov   & \Av_{22}
        \end{bmatrix} = \sum_{i \in [m]}
        \begin{bmatrix}
            \Pv_{i1} \\ \Pv_{i2} \\ \Pv_{i3} \\ \Pv_{i4}
        \end{bmatrix} \Rv_i
        \begin{bmatrix}
            \Qv_{i1} \\ \Qv_{i2} \\ \Qv_{i3} \\ \Qv_{i4}
        \end{bmatrix}^\top
    \end{align}
    其中$\Rv_i$只由$\Av_{11}, \Av_{12}, \Av_{21}, \Av_{22}$进行加减运算得到且$\Pv_{i1}, \ldots,\Pv_{i4}, \Qv_{i1}, \ldots, \Qv_{i4} \in \{ \pm \Iv, \zerov \}$。}则
\begin{align*}
    \begin{bmatrix}
        \Av_{11} \Bv_{11} + \Av_{12} \Bv_{21} \\
        \Av_{11} \Bv_{12} + \Av_{12} \Bv_{22} \\
        \Av_{21} \Bv_{11} + \Av_{22} \Bv_{21} \\
        \Av_{21} \Bv_{12} + \Av_{22} \Bv_{22}
    \end{bmatrix} = \sum_{i \in [m]}
    \begin{bmatrix}
        \Pv_{i1} \\ \Pv_{i2} \\ \Pv_{i3} \\ \Pv_{i4}
    \end{bmatrix} \Rv_i
    \green{\begin{bmatrix}
                   \Qv_{i1} \\ \Qv_{i2} \\ \Qv_{i3} \\ \Qv_{i4}
               \end{bmatrix}^\top
        \begin{bmatrix}
            \Bv_{11} \\ \Bv_{12} \\ \Bv_{21} \\ \Bv_{22}
        \end{bmatrix}} = \sum_{i \in [m]}
    \begin{bmatrix}
        \Pv_{i1} \\ \Pv_{i2} \\ \Pv_{i3} \\ \Pv_{i4}
    \end{bmatrix} \Rv_i \green{\Sv_i} = \sum_{i \in [m]}
    \begin{bmatrix}
        \Pv_{i1} \\ \Pv_{i2} \\ \Pv_{i3} \\ \Pv_{i4}
    \end{bmatrix} \Tv_i
\end{align*}
其中$\Sv_i = \Qv_{i1} \Bv_{11} + \Qv_{i2} \Bv_{12} + \Qv_{i3} \Bv_{21} + \Qv_{i4} \Bv_{22}$只由$\Bv_{11}, \Bv_{12}, \Bv_{21}, \Bv_{22}$进行加减运算得到。计算全部$m$个$\Tv_i = \Rv_i \Sv_i$会产生$m$个子问题。又$\Pv_{i1}, \ldots, \Pv_{i4} \in \{ \pm \Iv, \zerov \}$,因此
\begin{align*}
    \begin{bmatrix}
        \Av_{11} \Bv_{11} + \Av_{12} \Bv_{21} \\
        \Av_{11} \Bv_{12} + \Av_{12} \Bv_{22} \\
        \Av_{21} \Bv_{11} + \Av_{22} \Bv_{21} \\
        \Av_{21} \Bv_{12} + \Av_{22} \Bv_{22}
    \end{bmatrix} = \sum_{i \in [m]}
    \begin{bmatrix}
        \Pv_{i1} \\ \Pv_{i2} \\ \Pv_{i3} \\ \Pv_{i4}
    \end{bmatrix} \Tv_i =
    \begin{bmatrix}
        \Pv_{11} \Tv_1 + \cdots + \Pv_{m1} \Tv_m \\
        \Pv_{12} \Tv_1 + \cdots + \Pv_{m2} \Tv_m \\
        \Pv_{13} \Tv_1 + \cdots + \Pv_{m3} \Tv_m \\
        \Pv_{14} \Tv_1 + \cdots + \Pv_{m4} \Tv_m
    \end{bmatrix}
\end{align*}
只由$\Tv_1, \ldots, \Tv_m$进行加减运算得到。综上,关键就是如何使式(\ref{eq: decomposition})中的$m < 8$。

下面给出一个$m = 7$的分解方法,首先去掉左上的$\Av_{11}$和右下的$\Av_{22}$
\begin{align*}
    \widetilde{\Av} -
                      &
    \begin{bmatrix}
        \Av_{11} & \zerov & \Av_{11} & \zerov \\
        \zerov   & \zerov & \zerov   & \zerov \\
        \Av_{11} & \zerov & \Av_{11} & \zerov \\
        \zerov   & \zerov & \zerov   & \zerov
    \end{bmatrix} -
    \begin{bmatrix}
        \zerov & \zerov & \zerov & \zerov \\ \zerov & \Av_{22} & \zerov & \Av_{22} \\ \zerov & \zerov & \zerov & \zerov \\ \zerov & \Av_{22} & \zerov & \Av_{22}
    \end{bmatrix} =
    \begin{bmatrix}
        \zerov & \zerov & \Av_{12} - \Av_{11} & \zerov \\ \zerov & \Av_{11} - \Av_{22} & \zerov & \Av_{12} - \Av_{22} \\ \Av_{21} - \Av_{11} & \zerov & \Av_{22} - \Av_{11} & \zerov \\ \zerov & \Av_{21} - \Av_{22} & \zerov & \zerov
    \end{bmatrix} \\
                      & =
    \begin{bmatrix}
        \zerov & \zerov & \zerov & \zerov \\ \zerov & \Av_{11} - \Av_{22} & \Av_{11} - \Av_{22} & \zerov \\ \zerov & \Av_{22} - \Av_{11} & \Av_{22} - \Av_{11} & \zerov \\ \zerov & \zerov & \zerov & \zerov
    \end{bmatrix} +
    \begin{bmatrix}
        \zerov & \zerov & \Av_{12} - \Av_{11} & \zerov \\ \zerov & \zerov & \Av_{22} - \Av_{11} & \Av_{12} - \Av_{22} \\ \Av_{21} - \Av_{11} & \Av_{11} - \Av_{22} & \zerov & \zerov \\ \zerov & \Av_{21} - \Av_{22} & \zerov & \zerov
    \end{bmatrix} \\
                      & = \begin{bmatrix}
                              \zerov & \zerov & \zerov & \zerov \\ \zerov & \Av_{11} - \Av_{22} & \Av_{11} - \Av_{22} & \zerov \\ \zerov & \Av_{22} - \Av_{11} & \Av_{22} - \Av_{11} & \zerov \\ \zerov & \zerov & \zerov & \zerov
                          \end{bmatrix} +
    \begin{bmatrix}
        \zerov & \zerov & \zerov              & \zerov              \\
        \zerov & \zerov & \Av_{22} - \Av_{12} & \Av_{12} - \Av_{22} \\
        \zerov & \zerov & \zerov              & \zerov              \\
        \zerov & \zerov & \zerov              & \zerov
    \end{bmatrix} +
    \begin{bmatrix}
        \zerov & \zerov & \Av_{12} - \Av_{11} & \zerov \\
        \zerov & \zerov & \Av_{12} - \Av_{11} & \zerov \\
        \zerov & \zerov & \zerov              & \zerov \\
        \zerov & \zerov & \zerov              & \zerov
    \end{bmatrix}                                                                                                                                                                                \\
                      & \qquad +
    \begin{bmatrix}
        \zerov              & \zerov              & \zerov & \zerov \\
        \zerov              & \zerov              & \zerov & \zerov \\
        \Av_{21} - \Av_{11} & \Av_{11} - \Av_{21} & \zerov & \zerov \\
        \zerov              & \zerov              & \zerov & \zerov
    \end{bmatrix} +
    \begin{bmatrix}
        \zerov & \zerov              & \zerov & \zerov \\
        \zerov & \zerov              & \zerov & \zerov \\
        \zerov & \Av_{21} - \Av_{22} & \zerov & \zerov \\
        \zerov & \Av_{21} - \Av_{22} & \zerov & \zerov
    \end{bmatrix}                                                                                                                                                                                \\
    \Longrightarrow ~ & \widetilde{\Av} =
    \begin{bmatrix}
        \Iv \\ \zerov \\ \Iv \\ \zerov
    \end{bmatrix} \Av_{11}
    \begin{bmatrix}
        \Iv \\ \zerov \\ \Iv \\ \zerov
    \end{bmatrix}^\top +
    \begin{bmatrix}
        \zerov \\ \Iv \\ \zerov \\ \Iv
    \end{bmatrix} \Av_{22}
    \begin{bmatrix}
        \zerov \\ \Iv \\ \zerov \\ \Iv
    \end{bmatrix}^\top +
    \begin{bmatrix}
        \zerov \\ \Iv \\ -\Iv \\ \zerov
    \end{bmatrix} (\Av_{11} - \Av_{22})
    \begin{bmatrix}
        \zerov \\ \Iv \\ \Iv \\ \zerov
    \end{bmatrix}^\top +
    \begin{bmatrix}
        \zerov \\ \Iv \\ \zerov \\ \zerov
    \end{bmatrix} (\Av_{12} - \Av_{22})
    \begin{bmatrix}
        \zerov \\ \zerov \\ -\Iv \\ \Iv
    \end{bmatrix}^\top                                                                                                                                                                                                \\
                      & \qquad +
    \begin{bmatrix}
        \Iv \\ \Iv \\ \zerov \\ \zerov
    \end{bmatrix} (\Av_{11} - \Av_{12})
    \begin{bmatrix}
        \zerov \\ \zerov \\ -\Iv \\ \zerov
    \end{bmatrix}^\top +
    \begin{bmatrix}
        \zerov \\ \zerov \\ \Iv \\ \zerov
    \end{bmatrix} (\Av_{11} - \Av_{21})
    \begin{bmatrix}
        -\Iv \\ \Iv \\ \zerov \\ \zerov
    \end{bmatrix}^\top +
    \begin{bmatrix}
        \zerov \\ \zerov \\ \Iv \\ \Iv
    \end{bmatrix} (\Av_{21} - \Av_{22})
    \begin{bmatrix}
        \zerov \\ \Iv \\ \zerov \\ \zerov
    \end{bmatrix}^\top
\end{align*}

\section{算法实现}

根据上面的分解易知计算
\begin{align*}
    \begin{bmatrix}
        \Sv_1 \\ \Sv_2 \\ \Sv_3 \\ \Sv_4 \\ \Sv_5 \\ \Sv_6 \\ \Sv_7
    \end{bmatrix} =
    \begin{bmatrix}
        \Iv    & \zerov & \Iv    & \zerov \\
        \zerov & \Iv    & \zerov & \Iv    \\
        \zerov & \Iv    & \Iv    & \zerov \\
        \zerov & \zerov & -\Iv   & \Iv    \\
        \zerov & \zerov & -\Iv   & \zerov \\
        -\Iv   & \Iv    & \zerov & \zerov \\
        \zerov & \Iv    & \zerov & \zerov
    \end{bmatrix}
    \begin{bmatrix}
        \Bv_{11} \\ \Bv_{12} \\ \Bv_{21} \\ \Bv_{22}
    \end{bmatrix} =
    \begin{bmatrix}
        \Bv_{11} + \Bv_{21} \\ \Bv_{12} + \Bv_{22} \\ \Bv_{12} + \Bv_{21} \\ \Bv_{22} - \Bv_{21} \\ -\Bv_{21} \\ \Bv_{12} - \Bv_{11} \\ \Bv_{12}
    \end{bmatrix}, \quad
    \begin{bmatrix}
        \Rv_1 \\ \Rv_2 \\ \Rv_3 \\ \Rv_4 \\ \Rv_5 \\ \Rv_6 \\ \Rv_7
    \end{bmatrix} =
    \begin{bmatrix}
        \Av_{11} \\ \Av_{22} \\ \Av_{11} - \Av_{22} \\ \Av_{12} - \Av_{22} \\ \Av_{11} - \Av_{12} \\ \Av_{11} - \Av_{21} \\ \Av_{21} - \Av_{22}
    \end{bmatrix}
\end{align*}
共会产生$10$次加减运算,计算$\Tv_1 = \Rv_1 \Sv_1, \ldots, \Tv_7 = \Rv_7 \Sv_7$共会产生$7$个子问题,最后计算
\begin{align*}
    \begin{bmatrix}
        \Av_{11} \Bv_{11} + \Av_{12} \Bv_{21} \\
        \Av_{11} \Bv_{12} + \Av_{12} \Bv_{22} \\
        \Av_{21} \Bv_{11} + \Av_{22} \Bv_{21} \\
        \Av_{21} \Bv_{12} + \Av_{22} \Bv_{22}
    \end{bmatrix}
     & =
    \begin{bmatrix}
        \Iv \\ \zerov \\ \Iv \\ \zerov
    \end{bmatrix} \Tv_1 +
    \begin{bmatrix}
        \zerov \\ \Iv \\ \zerov \\ \Iv
    \end{bmatrix} \Tv_2 +
    \begin{bmatrix}
        \zerov \\ \Iv \\ -\Iv \\ \zerov
    \end{bmatrix} \Tv_3 +
    \begin{bmatrix}
        \zerov \\ \Iv \\ \zerov \\ \zerov
    \end{bmatrix} \Tv_4 +
    \begin{bmatrix}
        \Iv \\ \Iv \\ \zerov \\ \zerov
    \end{bmatrix} \Tv_5 +
    \begin{bmatrix}
        \zerov \\ \zerov \\ \Iv \\ \zerov
    \end{bmatrix} \Tv_6 +
    \begin{bmatrix}
        \zerov \\ \zerov \\ \Iv \\ \Iv
    \end{bmatrix} \Tv_7 \\
     & =
    \begin{bmatrix}
        \Tv_1 + \Tv_5 \\ \Tv_2 + \Tv_3 + \Tv_4 + \Tv_5 \\ \Tv_1 - \Tv_3 + \Tv_6 + \Tv_7 \\ \Tv_2 + \Tv_7
    \end{bmatrix}
\end{align*}
共会产生$8$次加减运算。

综上,一共会产生$7$个子问题和$18$次加减运算,此时递推关系变成
\begin{align*}
    T(n) = 7 \cdot T(n/2) + \frac{18}{4} c_1 n^2
\end{align*}
设$n = 2^k$,则
\begin{align*}
    T(2^k)               & = 7^1 \cdot T(2^{k-1}) + \frac{18}{4} c_1 4^k         \\
    7^1 \cdot T(2^{k-1}) & = 7^2 \cdot T(2^{k-2}) + 7^1 \frac{18}{4} c_1 4^{k-1} \\
    7^2 \cdot T(2^{k-2}) & = 7^3 \cdot T(2^{k-3}) + 7^2 \frac{18}{4} c_1 4^{k-2} \\
                         & \vdots                                                \\
    7^{k-1} \cdot T(2^1) & = 7^k \cdot T(2^0) + 7^{k-1} \frac{18}{4} c_1 4^1
\end{align*}
注意$7^k = (2^{\lg 7})^k = (2^k)^{\lg 7} = n^{\lg 7} \approx n^{2.81}$,
累加可得
\begin{align*}
    T(n) = c_2 n^{\lg 7} + \frac{18}{4} c_1 4^k \frac{1-(7/4)^k}{1-(7/4)} = c_2 n^{\lg 7} + 6 c_1 (n^{\lg 7} - n^2) = \left( c_2 + 6 c_1 \right) n^{\lg 7} - 6 c_1 n^2
\end{align*}

\end{document}
