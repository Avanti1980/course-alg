\documentclass{ctexart}
\usepackage{avanti}

\begin{document}
\title{\bf{图上的匹配、覆盖、流、割}}
\author{张腾}
\date{}
\maketitle

设二部图$\Gcal = (\Vcal, \Ecal)$,其中$\Vcal = \Vcal_1 \uplus \Vcal_2$,$\Ecal \subseteq \Vcal_1 \times \Vcal_2$,$\delta(v)$为与点$v$相连的边的集合。

若边集$\Mcal$中任意两条边没有公共顶点,则称$\Mcal$为\blue{匹配}~(matching)。现对任意边$e$赋予一个\blue{非负整数}~$x_e$,则匹配满足对任意点$v$有$\sum_{e \in \delta(v)} x_e \le 1$。

若点集$\Ccal$使得每条边都至少有一个顶点属于$\Ccal$,则称$\Ccal$为\blue{覆盖}~(cover)。现对任意点$v$赋予一个\blue{非负整数}~$z_v$,则覆盖满足对任意边$(u,v)$有$z_u + z_v \ge 1$。

设$\Av = [a_{v,e}] \in \{ 0,1 \}^{|\Vcal| \times |\Ecal|}$是二部图$\Gcal$对应的\blue{关联矩阵},即$a_{v,e} = 1_{e \in \delta(v)}$,则
\begin{align*}
    \forall v \in \Vcal, \sum_{e \in \delta(v)} x_e \le 1 & \Longleftrightarrow \Av \xv \le \onev      \\
    \forall (u,v) \in \Ecal, ~ z_u + z_v \ge 1            & \Longleftrightarrow \Av^\top \zv \ge \onev
\end{align*}

\section{最大匹配}

所有匹配中势最大的称为\blue{最大匹配},求解最大匹配可形式化成
\begin{align} \label{eq: max-matching}
    \max_{\xv} ~ \{ \onev^\top \xv : \xv \in \Zbb_+^{|\Ecal|}, ~ \Av \xv \le \onev \}
\end{align}
由于$\xv$是整数向量,这是一个整数规划,难以直接求解,将离散集合$\Zbb_+^{|\Ecal|}$放松成连续集合$\Rbb_+^{|\Ecal|}$,可得线性规划
\begin{align} \label{eq: relax-max-matching}
    \max_{\xv} ~ \{ \onev^\top \xv : \xv \ge \zerov, ~ \Av \xv \le \onev \}
\end{align}
注意$\{ \xv \ge \zerov, ~ \Av \xv \le \onev \} \Longleftrightarrow [\Av; -\Iv] \xv \le [\onev; \zerov]$,由于二部图的关联矩阵必然是\href{https://avanti1980.github.io/notes-on-math/posts/matrix/TU-matrix.html}{全幺模矩阵},故$[\Av; -\Iv]$也是全幺模矩阵,又$[\onev; \zerov]$是整数向量,故凸多面体$\{ \xv \ge \zerov, ~ \Av \xv \le \onev \}$的\href{https://avanti1980.github.io/notes-on-math/posts/optimization/analysis/extreme-point.html}{极点}是整数向量。由于线性规划必然在极点处取最优,因此线性规划(\ref{eq: relax-max-matching})的最优解就是整数规划(\ref{eq: max-matching})的最大匹配。

上述将离散整数约束替换为连续实数约束的操作,其实是将可行域由匹配集合扩大成其\blue{凸包}:

\begin{theorem}
    记匹配$\Mcal$对应的向量为$\xv^{(\Mcal)}$,$\Pcal (\Gcal) \triangleq \conv \{ \xv^{(\Mcal_1)}, \xv^{(\Mcal_2)}, \ldots \}$,$\Qcal (\Gcal)$定义为:
    \begin{align*}
        \Qcal (\Gcal) = \{ \xv \mid \xv \ge \zerov, ~ \Av \xv \le \onev \} = \left\{ \xv \in \Rbb_+^{|\Ecal|} \mid \forall v \in \Vcal, \sum_{e \in \delta(v)} x_e \le 1 \right\}
    \end{align*}
    那么$\Pcal (\Gcal) = \Qcal (\Gcal)$。
\end{theorem}

\begin{proof}
    正向比较简单,对任意$\xv = \sum_{i \in [n]} \alpha^{(\Mcal_i)} \xv^{(\Mcal_i)} \in \Pcal(\Gcal)$,非负性是显然的,又
    \begin{align*}
        \forall v \in \Vcal, \sum_{e \in \delta(v)} x_e & = \sum_{e \in \delta(v)} \sum_{i \in [n]} \alpha^{(\Mcal_i)} x^{(\Mcal_i)}_e = \sum_{i \in [n]} \alpha^{(\Mcal_i)} \overbrace{\blue{\sum_{e \in \delta(v)} x^{(\Mcal_i)}_e}}^{\le 1} \le \sum_{i \in [n]} \alpha^{(\Mcal_i)} = 1
    \end{align*}
    因此$\xv \in \Qcal (\Gcal)$。

    反向较为麻烦,对任意$\xv \in \Qcal (\Gcal)$,记$\supp(\xv) = \{ e \in \Ecal \mid x_e > 0 \}$。下面对$|\supp(\xv)|$进行归纳,若$|\supp(\xv)| = 0$,则$\xv = \zerov$,对应零匹配;若$|\supp(\xv)| = 1$,即存在唯一的边$e$使得$0 < x_e \le 1$,其余分量均为零,显然这样的$\xv$可以表示成零匹配和单边匹配的凸组合。若$|\supp(\xv)| \ge 2$,分两种情况讨论:
    \begin{itemize}
        \item $\supp(\xv)$不是匹配,则$\supp(\xv)$包含长度$\ge 2$的路径,不妨就设为$v_1 \xrightarrow{e_1} v_2 \xrightarrow{e_2} v_3$,由于$x_{e_1}, x_{e_2} > 0$,故$x_{e_1}, x_{e_2} < 1$,否则$\sum_{e \in \delta(v_2)} x_e = x_{e_1} + x_{e_2} > 1$。记$\xv = [x_{e_1};x_{e_2};\tilde{\xv}]$、$\dv = [1;-1;\zerov]$,易知
              \begin{align*}
                  \xv - x_{e_1} \dv = \begin{bmatrix} x_{e_1} \\ x_{e_2} \\ \tilde{\xv} \end{bmatrix} - x_{e_1} \begin{bmatrix} 1 \\ -1 \\ \zerov \end{bmatrix} = \begin{bmatrix} 0 \\ x_{e_1} + x_{e_2} \\ \tilde{\xv} \end{bmatrix} \triangleq \xv_2, \quad \xv + x_{e_2} \dv = \begin{bmatrix} x_{e_1} \\ x_{e_2} \\ \tilde{\xv} \end{bmatrix} + x_{e_2} \begin{bmatrix} 1 \\ -1 \\ \zerov \end{bmatrix} = \begin{bmatrix} x_{e_1} + x_{e_2} \\ 0 \\ \tilde{\xv} \end{bmatrix} \triangleq \xv_1
              \end{align*}
              于是$x_{e_1} x_{e_2} \dv = x_{e_2} (\xv - \xv_2) = x_{e_1} (\xv_1 - \xv)$,从而
              \begin{align*}
                  \xv = \frac{x_{e_1}}{x_{e_1} + x_{e_2}} \xv_1 + \frac{x_{e_2}}{x_{e_1} + x_{e_2}} \xv_2 = \conv\{ \xv_1, \xv_2 \}
              \end{align*}
              注意$|\supp(\xv_1)| = |\supp(\xv_2)| = |\supp(\xv)| - 1$,由归纳假设知$\xv_1,\xv_2 \in \Pcal(\Gcal)$,于是$\xv \in \Pcal(\Gcal)$。

        \item $\supp(\xv)$是匹配,不妨设$\supp(\xv) = \{ e_1, e_2, e_3, \ldots, e_n \}$且$x_{e_1} \le x_{e_2} \le x_{e_3} \le \cdots \le x_{e_n}$,定义
              \begin{align*}
                  \Mcal_i \triangleq \{ e_i, e_{i+1}, \ldots, e_n \}, \quad \xv^{(\Mcal_i)} = [\underbrace{0; \ldots; 0}_{1:i-1}; \underbrace{1; 1; \ldots; 1}_{i:n}; \underbrace{0; \ldots; 0}_{n+1:|\Ecal|}], \quad i \in [n]
              \end{align*}
              则
              \begin{align*}
                  \xv & = \begin{bmatrix} x_{e_1} \\ x_{e_2} \\ x_{e_3} \\ \vdots \\ x_{e_n} \\ \zerov \end{bmatrix} = \begin{bmatrix} x_{e_1} \\ x_{e_1} \\ x_{e_1} \\ \vdots \\ x_{e_1} \\ \zerov \end{bmatrix} + \begin{bmatrix} 0 \\ x_{e_2} - x_{e_1} \\ x_{e_2} - x_{e_1} \\ \vdots \\ x_{e_2} - x_{e_1} \\ \zerov \end{bmatrix} + \begin{bmatrix} 0 \\ 0 \\ x_{e_3} - x_{e_2} \\ \vdots \\ x_{e_3} - x_{e_2} \\ \zerov \end{bmatrix} + \cdots \\
                      & = x_{e_1} \xv^{(\Mcal_1)} + (x_{e_2} - x_{e_1}) \xv^{(\Mcal_2)} + (x_{e_3} - x_{e_2}) \xv^{(\Mcal_3)}                                                                                                                                                                                                                                                                                                                        \\
                      & \qquad + \cdots + (x_{e_n} - x_{e_{n-1}}) \xv^{(\Mcal_n)} + (1 - x_{e_n}) \zerov \in \Pcal (\Gcal)
              \end{align*}
    \end{itemize}
\end{proof}

由定义$\Pcal (\Gcal) = \conv \{ \xv^{(\Mcal_1)}, \xv^{(\Mcal_2)}, \ldots \}$知$\Pcal (\Gcal)$的任意极点都是$\Gcal$的匹配,反过来结论也成立:

\begin{theorem} \label{thm: extreme-point-matching}
    $\Gcal$的任意匹配都是$\Pcal$的极点。
\end{theorem}

\begin{proof}
    对任意匹配$\Mcal$和非零向量$\dv$,不妨设$d_e \neq 0$,注意$x^{(\Mcal)}_e \in \{0, 1\}$,因此$x^{(\Mcal)}_e \pm \epsilon d_e$总有一个不属于$[0,1]$,即$\xv^{(\Mcal)} \pm \epsilon \dv$总有一个不属于$\Pcal$,故$\xv^{(\Mcal)}$是$\Pcal$的极点。
\end{proof}

\section{完美匹配}

若匹配$\Mcal^\star$使得在子图$(\Vcal, \Mcal^\star)$中,所有点都有且仅有一条相连的边,则称为\blue{完美匹配}~(perfect matching)。完美匹配可表示为向量$\xv \in \Zbb_+^{|\Ecal|}$满足对任意$v \in \Vcal$有$\sum_{e \in \delta(v)} x_e = 1$,显然完美匹配是匹配的真子集。

\begin{theorem}
    设$\Pcal^\star (\Gcal)$为$\Gcal$的所有完美匹配构成的凸包,$\Qcal^\star (\Gcal)$定义为:
    \begin{align*}
        \Qcal^\star (\Gcal) = \{ \xv \mid \xv \ge \zerov, ~ \blue{\Av \xv = \onev} \} = \left\{ \xv \in \Rbb_+^{|\Ecal|} \mid \forall v \in \Vcal, \blue{\sum_{e \in \delta(v)} x_e = 1} \right\}
    \end{align*}
    则$\Pcal^\star (\Gcal) = \Qcal^\star (\Gcal)$。
\end{theorem}

\begin{proof}
    一方面,对任意$\xv = \sum_{i \in [n]} \alpha^{(\Mcal_i^\star)} \xv^{(\Mcal_i^\star)} \in \Pcal^\star(\Gcal)$,易知
    \begin{align*}
        \forall v \in \Vcal, \sum_{e \in \delta(v)} x_e & = \sum_{e \in \delta(v)} \sum_{i \in [n]} \alpha^{(\Mcal_i^\star)} x^{(\Mcal_i^\star)}_e = \sum_{i \in [n]} \alpha^{(\Mcal_i^\star)} \sum_{e \in \delta(v)} x_e^{(\Mcal_i^\star)} = \sum_{i \in [n]} \alpha^{(\Mcal_i^\star)} = 1
    \end{align*}
    因此$\xv \in \Qcal^\star (\Gcal)$。

    另一方面,对任意$\xv \in \Qcal^\star(\Gcal) \subseteq \Qcal(\Gcal) = \Pcal(\Gcal)$,设$\xv = \sum_{i \in [n]} \alpha^{(\Mcal_i)} \xv^{(\Mcal_i)}$。用反证法,若其凸组合表示中存在不完美匹配$\Mcal_j$,设$v$不是$\Mcal_j$中边的顶点,则
    \begin{align*}
        \sum_{e \in \delta(v)} x_e = \sum_{e \in \delta(v)} \sum_{i \in [n] \setminus \{j\}} \alpha^{(\Mcal_i)} x_e^{(\Mcal_i)} = \sum_{i \in [n] \setminus \{j\}} \alpha^{(\Mcal_i)} \sum_{e \in \delta(v)} x_e^{(\Mcal_i)} \le \sum_{i \in [n] \setminus \{j\}} \alpha^{(\Mcal_i)} < 1
    \end{align*}
    这和$\Qcal^\star (\Gcal)$的定义矛盾,故$\xv$的凸组合表示中不存在不完美匹配,即$\xv \in \Pcal^\star (\Gcal)$。
\end{proof}

\begin{theorem} \label{thm: extreme-point-perfect-matching}
    $\Gcal$的任意完美匹配都是$\Pcal^\star$的极点。
\end{theorem}

\begin{proof}
    完美匹配也是匹配,因此是$\Pcal$的极点,故无法由$\Pcal$中其它点的凸组合表示,又$\Pcal^\star \subseteq \Pcal$,因此也无法由$\Pcal^\star$中其它点的凸组合表示,从而也是$\Pcal^\star$的极点
\end{proof}

对于完全二部图$\Kcal_{n,n}$有$|\Ecal| = n^2$,对任意$\xv \in \Qcal^\star(\Kcal_{n,n})$有
\begin{align*}
    \xv \in \Rbb_+^{n^2}, ~ \forall v \in \Vcal, \sum_{e \in \delta(v)} x_e = 1
\end{align*}
又每个点恰有$n$条相连的边,因此$\xv$也可以写成一个$n \times n$的\blue{双随机矩阵}~(所有行和、列和均为$1$)。另一方面,对于完美匹配$\Mcal$,每个点有且仅有一条相连的边,其对应的$\xv^{(\Mcal)}$可以写成置换矩阵(每行、每列有且仅有一个$1$,其余为零),由定理\ref{thm: extreme-point-perfect-matching}知\blue{双随机矩阵集合的极点是置换矩阵},这就是Birkhoff-von Neumann定理。

\section{König定理}

前文已述最大匹配问题可放松成线性规划
\begin{align*}
    \max_{\xv} ~ \{ \onev^\top \xv : \xv \ge \zerov, ~ \Av \xv \le \onev \}
\end{align*}
引入Lagrange乘子$\yv \in \Rbb_+^{|\Ecal|}$、$\zv \in \Rbb_+^{|\Vcal|}$,对偶函数$\Lcal(\xv, \yv, \zv) = \onev^\top \xv + \yv^\top \xv - \zv^\top (\Av \xv - \onev)$,易知
\begin{align*}
    \frac{\partial \Lcal}{\partial \xv} = \onev + \yv - \Av^\top \zv = \zerov \Longrightarrow \Av^\top \zv - \onev = \yv \geq \zerov
\end{align*}
故对偶问题为线性规划
\begin{align} \label{eq: relax-min-vertex-cover}
    \min_{\zv} ~ \{ \onev^\top \zv : \zv \ge \zerov, ~ \Av^\top \zv \ge \onev \}
\end{align}
显然这是将\blue{最小点覆盖}问题
\begin{align} \label{eq: min-vertex-cover}
    \min_{\zv} ~ \{ \onev^\top \zv : \zv \in \Zbb_+^{|\Vcal|}, ~ \Av^\top \zv \ge \onev \}
\end{align}
的离散集合$\Zbb_+^{|\Vcal|}$放松成连续集合$\Rbb_+^{|\Ecal|}$得到的线性规划。同理由$\{ \zv \ge \zerov, ~ \Av^\top \zv \ge \onev \} \Longleftrightarrow [-\Av^\top; -\Iv] \zv \le [-\onev; \zerov]$以及$\Av$是全幺模矩阵知凸多面体$\{ \zv \mid \zv \ge \zerov, ~ \Av^\top \zv \ge \onev \}$的极点是整数向量。由于线性规划必然在极点处取最优,因此线性规划(\ref{eq: relax-min-vertex-cover})的最优解就是整数规划(\ref{eq: min-vertex-cover})的最小点覆盖。

综上,\blue{最大匹配、最小点覆盖这两类整数规划问题,其最优解就是将整数约束放松后导出的线性规划的最优解,且这两类相应的线性规划互为对偶问题}。

\begin{theorem} [König]
    对于二部图$\Gcal = (\Vcal, \Ecal)$,设最大匹配问题的最优值为$\maxm(\Gcal)$,最小点覆盖问题的最优值为$\minvc(\Gcal)$,则有$\maxm(\Gcal) = \minvc(\Gcal)$。
\end{theorem}

\begin{proof}
    $\minvc(\Gcal) \ge \maxm(\Gcal)$是显然的,因为对最大匹配中的任意一条边,至少要覆盖其中一个顶点。

    下面证明另一个方向,若$\Ecal = \emptyset$,则$\maxm(\Gcal) = \minvc(\Gcal) = 0$,故不妨设$\Ecal$非空。对$|\Vcal|$进行归纳,若$|\Vcal| = 2$,易知$\maxm(\Gcal) = \minvc(\Gcal) = 1$。若$|\Vcal| > 2$,设$\zv^\star$是最小点覆盖问题的最优解,由于存在点$v$使得$z_v^\star > 0$,故根据\blue{互补松弛条件}可得
    \begin{align*}
        z_v^\star \left( \sum_{e \in \Ecal} a_{v,e} x_e^\star - 1 \right) = 0 \Longrightarrow 1 = \sum_{e \in \Ecal} a_{v,e} x_e^\star = \sum_{e \in \delta(v)} x_e^\star
    \end{align*}
    注意$\xv^\star$是最大匹配,故$v$出现在所有的最大匹配中,记$\tilde{\Gcal}$为$\Gcal$删除点$v$及其相连边后得到的图,于是
    \begin{align*}
        \maxm(\tilde{\Gcal}) = \maxm(\Gcal) - 1
    \end{align*}
    由归纳假设知$\maxm(\tilde{\Gcal}) = \minvc(\tilde{\Gcal})$,于是
    \begin{align*}
        \minvc(\Gcal) & \le \minvc(\tilde{\Gcal}) + 1 \\
                      & = \maxm(\tilde{\Gcal}) + 1    \\
                      & = \maxm(\Gcal)
    \end{align*}
\end{proof}

König定理还可进一步推广,设$b$-匹配对应的向量满足对任意点$v$有$\sum_{e \in \delta(v)} x_e \le b_v$;$c$-点覆盖对应的向量满足对任意边$e = (u,v)$有$z_u + z_v \ge c_e$,易知有
\begin{align*}
    \max_{\xv} ~ \{ \cv^\top \xv : \xv \ge \zerov, ~ \Av \xv \le \bv \} = \min_{\zv} ~ \{ \bv^\top \zv : \zv \ge \zerov, ~ \Av^\top \zv \ge \cv \}
\end{align*}
即\blue{最大$c$-加权$b$-匹配等于最小$b$-加权$c$-点覆盖}。

\section{最大流与最小割}

类似于最大匹配和最小点覆盖,最大流和最小割也是一组对偶问题。给定有向流网络$\Gcal = (\Vcal, \Ecal)$、源点$s$、汇点$t$,设$\delta_{\text{in}}(v) / \delta_{\text{out}}(v)$是以点$v$为终点/起点的入边/出边集合,$\Av = [a_{v,e}] \in \{ 0, \pm 1 \}^{|\Vcal| \times |\Ecal|}$是$\Gcal$对应的关联矩阵,即
\begin{align*}
    a_{v,e} = \begin{cases}
                  1  & e \in \delta_{\text{in}} (v)  \\
                  -1 & e \in \delta_{\text{out}} (v) \\
                  0  & \ow
              \end{cases}
\end{align*}
$\tilde{\Av}$为$\Av$去掉$s$、$t$对应行的子矩阵,注意有向流网络中源点$s$只有出边、汇点$t$只有入边,因此$\tilde{\Av}$其实也是$\Gcal$删除$s$、$t$及其所有相连边后的有向图的关联矩阵,故$\tilde{\Av}$是全幺模矩阵。

最大流问题可形式化为线性规划:
\begin{align*}
    \max_{\xv} ~ \{ \av^\top \xv : \zerov \le \xv \le \cv, ~ \tilde{\Av} \xv = \zerov \}
\end{align*}
其中$\av^\top$是$\Av$中汇点$t$对应的行,$\zerov \le \xv \le \cv$约束流的上下界,$\tilde{\Av} \xv = \zerov$约束非源点、汇点的流量要守恒。注意
\begin{align*}
    \{ \xv \mid \zerov \le \xv \le \cv, ~ \tilde{\Av} \xv = \zerov \} \Longleftrightarrow [\tilde{\Av}; -\tilde{\Av}; \Iv; -\Iv] \xv \leq [\zerov; \zerov; \cv; \zerov]
\end{align*}
由$\tilde{\Av}$是全幺模矩阵知$[\tilde{\Av}; -\tilde{\Av}; \Iv; -\Iv]$也是全幺模矩阵,若流量上限$\cv$是整数向量,则可行域$\{ \zv \mid \zerov \le \xv \le \cv, ~ \tilde{\Av} \xv = \zerov \}$的极点也是整数向量,即最大流是整数流。

引入Lagrange乘子$\yv \in \Rbb_+^{|\Ecal|}$、$\zv \in \Rbb_+^{|\Ecal|}$、$\tilde{\wv} \in \Rbb_+^{|\Vcal|-2}$,对偶函数$\Lcal(\xv, \yv, \zv, \tilde{\wv}) = \av^\top \xv + \yv^\top \xv - \zv^\top (\xv - \cv) - \tilde{\wv}^\top \tilde{\Av} \xv$,易知
\begin{align*}
    \frac{\partial \Lcal}{\partial \xv} = \av + \yv - \zv - \tilde{\Av}^\top \tilde{\wv} = \zerov \Longrightarrow \tilde{\Av}^\top \tilde{\wv} + \zv \ge \av
\end{align*}
故对偶问题为
\begin{align*}
    \min_{\tilde{\wv}, \zv} ~ \{ \cv^\top \zv : \zv \ge \zerov, ~ \tilde{\Av}^\top \tilde{\wv} + \zv \ge \av \}
\end{align*}
注意
\begin{align*}
    \{ \zv \mid \zv \ge \zerov, ~ \tilde{\Av}^\top \tilde{\wv} + \zv \ge \av \} \Longleftrightarrow [-\tilde{\Av}^\top, -\Iv; \zerov, -\Iv] [\tilde{\wv}; \zv] \leq [-\av; \zerov]
\end{align*}
由$\tilde{\Av}$是全幺模矩阵知$[-\tilde{\Av}^\top, -\Iv; \zerov, -\Iv]$也是全幺模矩阵,故对偶问题的最优解$\tilde{\wv}^\star$、$\zv^\star$也是整数向量。

注意$\tilde{\wv}^\star$的维度为$|\Vcal| - 2$,与$\tilde{\Av}$的行对应,现添加$w_s^\star = 0$、$w_t^\star = -1$将其扩充为$\wv^\star$,与$\Av$的行对应,于是$\Av^\top \wv^\star + \zv^\star = \tilde{\Av}^\top \tilde{\wv}^\star - \av + \zv^\star \ge \zerov$,从而$\zv^\star = \max \{ \zerov, - \Av^\top \wv^\star \}$,即对任意边$e = (u,v)$有$z^\star_e = \max \{ 0, w_u^\star - w_v^\star \}$。

定义$\Scal = \{ v \in \Vcal \mid w_v^\star \ge 0 \}$,$\overline{\Scal} = \Vcal \setminus \Scal$,显然$s \in \Scal$、$t \in \overline{\Scal}$,将边分为四类:
\begin{itemize}
    \item $\delta(\Scal) \triangleq \{ (u,v) \in \Ecal \mid u \in \Scal, ~ v \in \Scal \}$为所有起点、终点均属于$\Scal$的边的集合;
    \item $\delta(\overline{\Scal}) \triangleq \{ (u,v) \in \Ecal \mid u \in \overline{\Scal}, ~ v \in \overline{\Scal} \}$为所有起点、终点均属于$\overline{\Scal}$的边的集合;
    \item $\delta_{\text{out}}(\Scal) \triangleq \{ (u,v) \in \Ecal \mid u \in \Scal, ~ v \in \overline{\Scal} \}$为所有起点属于$\Scal$、终点属于$\overline{\Scal}$的边的集合;
    \item $\delta_{\text{in}}(\Scal) \triangleq \{ (u,v) \in \Ecal \mid u \in \overline{\Scal}, ~ v \in \Scal \}$为所有起点属于$\overline{\Scal}$、终点属于$\Scal$的边的集合;
\end{itemize}
注意在将所有$\delta_{\text{out}}(\Scal)$中的边删除后,$s$、$t$不再连通,因此$\delta_{\text{out}}(\Scal)$称为割(cut)。

由于$w_v^\star$都是整数,因此对任意边$e = (u,v) \in \delta_{\text{out}}(\Scal)$有$z_e^\star \ge w_u^\star - w_v^\star \ge 1$,于是
\begin{align*}
    \cv^\top \zv^\star \ge \sum_{e \in \delta_{\text{out}}(\Scal)} c_e z_e^\star \ge \sum_{e \in \delta_{\text{out}}(\Scal)} c_e \ge \sum_{e \in \delta_{\text{out}}(\Scal)} x_e^\star \ge \sum_{e \in \delta_{\text{in}}(t)} x_e^\star = \av^\top \xv^\star
\end{align*}
其中第一个不等号是因为$z_e^\star \ge 0$;第二个不等号是因为对任意边$e \in \delta_{\text{out}}(\Scal)$有$z_e^\star \ge 1$;第三个不等号是因为$c_e$是边$e$的流量上限;第四个不等号是因为$\delta_{\text{out}}(\Scal)$上的流量未必会全部进入汇点,可能会有一部分通过$\delta_{\text{in}}(\Scal)$再折回$\Scal$。

根据强对偶定理所有的不等号都取等号,由此可以得到一些有趣的结论:
\begin{itemize}
    \item 根据第一个不等号取等号,对任意边$e \not \in \delta_{\text{out}}(\Scal)$有$z_e^\star = 0$,即对任意$\delta(\Scal)$、$\delta(\overline{\Scal})$、$\delta_{\text{in}}(\Scal)$中的边$e$,都有$z_e^\star = 0$;
    \item 根据第二个不等号取等号,对任意边$e = (u,v) \in \delta_{\text{out}}(\Scal)$有$z_e^\star = 1$,故只可能是$w_u^\star = 0$、$w_v^\star = -1$,于是对任意边$e = (p, u) \in \delta(\Scal)$,必然有$w_p^\star = 0$,否则$z_e^\star \ge w_p^\star - w_u^\star > 0$,与前一个结论矛盾,依此类推,对所有$\Scal$中的点$u$都有$w_u^\star = 0$。同理,对所有$\overline{\Scal}$中的点$v$都有$w_v^\star = -1$;
    \item 根据第三个不等号取等号,当流量达到最大时,$\delta_{\text{out}}(\Scal)$中每条边的流量都达到上限,这个也可由互补松弛条件$z_e (x_e - c_e) = 0$得到:$z_e^\star = 1 > 0 \Longrightarrow x_e^\star = c_e$;
    \item 根据第四个不等号取等号,$\delta_{\text{out}}(\Scal)$上的流量全部进入$t$,不折回$\Scal$,即$\delta_{\text{in}}(\Scal)$上的流量为零,这个也可由互补松弛条件$y_e x_e = 0$得到:$z_e^\star = 0 > -1 = w_u^\star - w_v^\star$,故$y_e^\star = z_e^\star - (w_u^\star - w_v^\star) > 0$,从而$x_e^\star = 0$。
\end{itemize}



\end{document}

